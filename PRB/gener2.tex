\documentstyle[aps,prb]{revtex}
\title{Generation of phonon bursts by strong local vibration in ionic crystals}
\author{V.Hizhnyakov $^{a}$, D.Nevedrov $^{a,b}$}
\address{$^{a}$Institute of Theoretical Physics,
University of Tartu, T\"{a}he 4, Tartu EE2400, Estonia.\\ 
$^{b}$NORDITA, Blegdamsvej 17, Copenhagen DK-2100, Denmark.\\
e-mail: hizh@fi.tartu.ee}
\begin{document}
\maketitle
\begin{center}

\end{center}

\begin{abstract}
Quantum decay of a strongly excited local mode anharmonically interacting with
phonons is examined on the basis of the recently developed nonperturbative 
theory \cite{hizhrev}. The main goal is to study relaxation jumps of the mode,
which have been predicted by the theory, in real systems. Concretely the 
relaxation laws of the odd local modes associated with light impure or host 
ions in alkali halide crystals are calculated. It is found that for amplitudes
$\gtrsim 0.3 \AA$ a number of relaxation jumps take places leading to 
generation of phonon bursts. Although the jumps depend on the amplitude and 
the direction of the vibration, on the whole emission of phonons by the mode 
resembles a sharp explosion. This may open new ways in generation of ultrashort
phonon pulses.  
\end{abstract}

\section{Introduction}
The nonlinear dynamics of strong vibrational excitations in crystals has 
attracted remarkable attention during the last years. The opportunities of 
applying novel numerical techniques play an essential role in the 
investigation of nonlinear discrete systems, allowing the observation of
different  phenomena. One of the effects, - the existence of long-living
localized vibrations in a pure anharmonic lattice (so-called intrinsic local
modes (ILM) or self-localized solitons (SLS)) \cite{dolgov,sivtak} is of 
particular interest, as it links local lattice dynamics with physics of 
solitons and underlines in general the importance of strongly excited local 
modes. So far the research of strong anharmonic effects in lattice dynamics 
as well as molecular dynamics simulations in nonlinear lattice has been 
carried out within the frame of classical mechanics. Quantum effects, which 
are of principal importance here as they lead to extra mechanisms in the decay
of vibrational excitations,
have been discussed only in frames of standard perturbation theory (see, e.g.
\cite{ovchi}). However in many cases the mentioned theory
cannot be used since one cannot assume that the perturbation parameter is 
small (note that one of the conditions of existence of SLSs in a nonlinear 
lattice is large energy of the vibration). E.g. in crystals, considered
below this theory would give  $\sim \hbar \omega_l^2$ for the relaxation rate 
of a mode with energy $E_l \sim 10 \hbar \omega_l$ (corresponding amplitude 
$A\sim 0.3 \AA$), which violates the condition of applicability of the theory.
Therefore the perturbation theory is applicable only at $E_l\ll E_{cr}\sim
10\hbar\omega_l$. 

A nonperturbative theory of the two-phonon anharmonic decay of a strong local 
mode \cite{hizhrev} associated with a substitutional impurity was recently 
proposed. Extension of the theory to the quantum decay of the
SLSs in the perfect monatomic chain was presented in \cite{chain}. One of 
the results of the theory is an explosion-like release of energy of the local 
vibration if the amplitude of the mode reaches some specific critical value. 
In \cite{multi} three- and more-phonon processes behave been also 
accounted and found to lead to an analogous effect. Such a behavior of the 
multiphonon processes is a consequence of a mutual interplay between strong 
anharmonic and quantum effects. An experimental support for relaxation jumps 
due to multiphonon anharmonic processes had recently come from study of hot 
luminescence of self-trapped excitons in solid $Xe$ under laser
excitation.  

The method, used in \cite{hizhrev,procest,hizhnev} is based on the 
consideration of the effect of the strongly excited local mode to phonon 
operators. This effect was found to be, in general analogous to that of 
gravitationally collapsing star (black hole) to field operators, found by 
Hawking \cite{hawking}: time-dependent destruction operators become a linear 
combination of the initial destruction and creation operators. Therefore the 
initial zero-point state is not the zeroth state for the time-dependent 
destruction operators, i.e. there appear phonons the number of which increases
in time. The energy of generated phonons comes from the energy of the local 
vibration causing dumping (relaxation) of this vibration. This approach to 
description of relaxation is applicable for arbitrary (also for large) 
amplitude (energy) of the local mode, as well as  Hawking theory 
\cite{hawking} is applicable to quantum emission of black holes with arbitrary 
mass (within the Planck mass limit). Moreover generation of particles in 
both cases has a common feature: explosion-like release of the energy at 
definite stages of the process (the details as well as the basic equations, 
which describe the evolution, however are essentially different). Note also 
that the approach works  both, in the case of classical 
\cite{hizhrev,procest,hizhnev}, as well as quantum-mechanical description 
of the local mode. 

In this communication the theory \cite{procest,hizhrev,hizhnev} is extended to
the ILM's associated with light host atoms in three-dimensional crystals and 
is applied to alkali halide crystals, where 
all parameters of the theory are known. The strongly excited local modes are 
considered classically, while initially not excited phonons, -
quantum-mechanically. Relaxation laws of odd local vibrations associated with  
strongly vibrating $F^-$ and $Na^+$ ions in these crystal are 
carried through. Spectra of phonons are calculated in the shell model; 
anharmonic constants are determined from the Born-Coulomb-Mayer potentials. 
Both central and noncentral forces are taken into account. Local dynamics is 
described by the Lifshits Green's function method with account of anharmonic 
renormalization of the local force constants. It is shown that the relaxation 
laws at large amplitudes of the local mode are strongly nonexponential: at 
very large amplitudes relaxation is rather slow being increased within the
course of relaxation. At critical amplitudes $\gtrsim 0.3 \AA$ the 
relaxation rate sharply increases, and then drops causing a relaxation jump, 
after which it starts to increase again till the next critical point. The 
position and the number of the jumps depends on the crystal and on the 
impurity as well as on the amplitude and the direction of the vibrations in 
the crystal lattice. On the whole large part of energy of the local mode is 
released in rather short time of the order of few picoseconds 
and the emission of phonons by the mode resembles an sharp 
explosion. The final  stage of relaxation of a mode associated with
impurity is exponential while the ILM does not have at all the usual,
exponential stage (total life-time of ILMs is finite). We 
demonstrate also that the process of relaxation is different for different 
directions of vibrations in the crystal lattice; depending on 
the initial conditions, vibrations in [100] or [111] directions appear to be 
more stable. These results, in our opinion, may be of importance for the 
understanding of the nature of defect formation by high-energy particles and 
by electronic excitations in solids, it can be also relevant for the 
description of chemical reactions. They also may open new ways in 
generation of ultrashort phonon  pulses.

%GENERAL
\section{General}
Let us consider a vibrational system with the potential energy operator
\begin{equation}
\hat{V} = \frac{1}{2} \sum_{n_{1} n_{2}}\!{V^{(2)}_{n_{1} n_{2}}
\hat{U}_{n_{1}} \hat{U} _{n_{2}}}
+ \frac{1}{3}\! \sum_{n_{1} n_{2} n_{3}}\!\!{V^{(3)}_{n_{1} n_{2} n_{3}}
\hat{U}_{n_{1}} \hat{U}_{n_{2}} \hat{U}_{n_{3}}} + \ldots\,,
\label{eq:v}
\end{equation}
where $\hat{U}_{n}$ are the operators of Cartesian displacements of atoms 
situated at the site $n$, $V^{(2)},\, V^{(3)},\,\ldots$ are harmonic 
and anharmonic springs. The operators $\hat{U}_{n}$ satisfy the 
following equations of motion
\begin{equation}
-M_{n} \frac{\partial^{2} \hat{U}_{n}}{\partial t^{2}} = 
\sum_{n_{1}}{V^{(2)}_{n n_{1}} \hat{U}_{n_{1}}} + 
\sum_{n_{1} n_{2}}{V^{(3)}_{n_{1} n_{2} n_{3}} \hat{U}_{n_{1}}
\hat{U}_{n_{2}}} + \ldots\,, 
\label{eq:motion}
\end{equation}
here $M_{n}$ are the masses of atoms.

We suppose that a local mode (ILM or local mode induced by impurity) is
strongly excited at the time moment $t=0$ at the site $n=0$ and its nearest
neighbors. To take account of this excitation we present the operators 
$\hat{U}_{n}$ in the form $\hat{U}_{n} = A_{n}(t) + \hat{q}_{n}/sqrt{M_n}$
where $\hat{q}_{n}$ stands for the operators of a reduced coordinate, $A_{n}$ 
are strong "classical" amplitudes which satisfy the same equations 
(\ref{eq:motion}) but with $A_{n}$ instead of $\hat{U}_{n}$. These are 
standard equations of classical local dynamics. As it is known \cite{dolgov}, 
the solution describing a local mode with the frequency $\omega_{l}$ has the 
form $A_{n}(t) = A_{n} \cos{\omega_{l} t} + \xi_{n} + O(\omega_l)$, where
$O(\omega_l)$ contains a combination of small order terms of frequencies
$3\omega_{l},\, 5\omega_{l},\, \ldots$ (these terms are neglected below). The 
amplitudes $A_{n}$ and the shifts $\xi_{n}$ can be found by equating the
coefficients before the terms with the same time dependence 
\cite{dolgov,sivtak}. The corresponding equations read
\begin{eqnarray}
M_{n} \omega_{l}^{2} C_{n} &=& \sum_{n1}{V^{(2)}_{n n_{1}} C_{n_{1}}}
+ 2\,\sum_{n_{1} n_{2}}{V^{(3)}_{n n_{1} n_{2}} C_{n_{1}} \xi_{n_{2}}} +
\nonumber \\
&+& 3\sum_{n n_{1} n_{2} n_{3}}{V^{(4)}_{n n_{1} n_{2} n_{3}}
{\Big (}\frac{1}{4}C_{n_{1}} C_{n_{2}} + \xi_{n_{1}} \xi_{n_{2}}{\Big )}
C_{n_{3}}} + \ldots\,, \nonumber
\end{eqnarray}
\begin{eqnarray}
\sum_{n_{1}}{V^{(2)}_{n n_{1}} \xi_{n_{1}}} +
\sum_{n_{1} n_{2}}{V^{(3)}_{n n_{1} n_{2}} {\Big (}\frac{3}{2} C_{n_{1}}
C_{n_{2}} +
\xi_{n_{1}}
\xi_{n_{2}}{\Big )}} + \ldots = 0\,.\nonumber
\end{eqnarray}
%TRANSFORMATION OF PHONON OPERATORS
\subsection{Transformation of phonon operators in time}
The quantum effects lead to the the energy 
loss of the local mode and to the variation of $A_{n}$ and $\xi_{n}$ in time. 
We assume that these variations are slow (i.e. small for the characteristic 
time scale $\omega_{l}^{-1}$). The goal is to find the rate of these 
variations. To describe the time evolution (decay) of the local mode 
we consider the equations for $\hat{q}_{n}$ (they follow from (2):
\begin{equation}
\frac{d^{2} \hat{q}_{n}}{d t^{2}} = 
\sum_{n'}{{\big (}V_{2 n n'}+\tilde{V}_{2 n n'}+ 
W_{n n'}(t){\big )}\hat{q}_{n'}}\,,
\label{eq:mdx}
\end{equation}
where $V_{2 n n'}$ is the dynamical matrix in harmonic approximation,
\begin{eqnarray}
\tilde{V}_{2 n n'} & = & \frac{2}{\sqrt{M_{n} M_{n'}}}
\sum_{n'}{V^{(3)}_{n n' n_1}\xi_{n_1}} + \nonumber \\
& + & \frac{3}{4} \sum_{n_{1} n_{2}}{V^{(4)}_{n n'n_{1} n_{2}}{\big
(}A_{n_{1}} A_{n_{2}} + 2\xi_{n_{1}} \xi_{n_{2}}{\big )}}{\Big )} 
+ \ldots\, 
\end{eqnarray} 
describes anharmonic renormalization of the dynamic matrix 
due to strong local vibration,
\begin{eqnarray}
W_{n n'}(t)  & = & 2\,v_{n n'}\Theta(t) \cos{\omega_{l} t} ,
\end{eqnarray}
\begin{eqnarray}
v_{n n'} & = & \frac{A}{\sqrt{M_{n} M_{n'}}}
\sum_{n'}{V^{(3)}_{nn'n_{1}} a_{n_{1}}}+\ldots\,
\end{eqnarray}
is the anharmonic relaxation matrix; $\Theta (t)$ is the Heaviside 
step-wise function: $\Theta(t) =1$, if $t>0$, $\Theta(t) = 0$, if $t<0$, 
$a_n=A_n/A$, $A$ is the amplitude of the local mode.
In equation (\ref{eq:mdx}) we neglect the contribution of terms of higher
order with respect to $\hat{q}_{n}$ . The terms included into consideration 
stand for two-phonon decay, while the neglected terms take account of three-,
four- etc. phonon processes; the latter processes are examined in 
\cite{zhizh}.

Let us now introduce the phonon operators 
$\hat{q}_{i}=\sum_{n}{e_{n_{i}}\hat{q}_{n}}$ with the elements $e_{ni}$ of 
the orthogonal matrix $e$ satisfying the conditions
\begin{equation}
\omega_{i}^{2} e_{n_{i}}  =  
\sum_{n'}{e_{n'_{i}} \frac{\tilde{V}_{2nn'}}{\sqrt{M_{n}M_{n'}}}}, 
\quad
\sum_{i}{e_{n_{i}} e_{n'_{i}}}  =  \delta_{nn'}.
\label{eq:om}
\end{equation}
these elements, as well as the frequencies of the phonons $\omega_{i}$, can
be found by standard methods of lattice dynamics \cite{marad}.
This allows one to diagonalize the time-independent part of the phonon
Hamiltonian: 
$H_{0_{ph}}=\sum_i (-\partial^2/\partial\hat{q}_i ^2+
\omega_i^2\hat{q}_i^2)/2$.
The remained time-dependent (oscillatory) part of the phonon hamiltonian
($\sim W_{nn'}(t)$, see (5),(6)) causes changes  of $\hat{q}_i$ in time.
To find these changes we present $\hat{q}_i(t)$ in the form
$\hat{q}_{i}(t)= \sqrt{\hbar/2\omega_{i}}(g_{i}(t) \hat{a}_{i} + 
g_{i}^{*}(t) \hat{a}_{i}^{+})$ where 
$\hat{a}_{i}$ and $\hat{a}_{i}^{+}$ are the destruction and the creation
phonon operators at the initial time moment (when $W=0$). The functions 
$g_{i}(t)$ satisfy the set of classical equations of motion,
\begin{equation}
\ddot{g}_{i} + \omega_{i}^{2} g_{i} = - \omega_{i}
\sum_{i'}{{\big (}\bar{e}_{i} W(t) \bar{e}_{i'}{\big )}g_{i'}}
\label{eq:mot}
\end{equation}
($\bar{e}_{i} = e_{i}/\sqrt{\omega_{i}}$) with the "nonclassical" initial 
conditions $g_{i}(t) = e^{i\varphi_{i} - i\omega_{i} t}$, $t \leq 0$ 
$\varphi_{i}$ being random phases. The corresponding set of integral equations
for $g_{i}$ reads:
\begin{eqnarray}
g_{i}(t) & = & e^{-i\omega_{i} t + i\varphi_{i}} - \nonumber \\ 
& - & \int_{0}^{t}\!{d\tau \,\sin{\omega_{i}(t\!-\!\tau )}\,
\sum_{i'}{{\big (}\bar{e}_{i}\, W(\tau)\, \bar{e}_{i'}{\big
)}\,g_{i'}(\tau)}}\,.
\label{eq:g}
\end{eqnarray}
One can see that for $t\gg \omega_i^{-1}$ the time-dependence of $g_i$ is
of the form $g_i(t) =\mu_i e^{-i\omega_i t} + \nu_i e^{(i\omega_i t)}$ with
$\mu_i$ and $\nu_i$ satisfying the equations
\begin{equation}
\mu_i(t) \simeq e^{i\varphi_i} +\frac{1}{2i} \int_0^t d\tau 
\sum_{i'}(e_i W(\tau) e_{i'}) e^{i(\omega_i - \omega_{i'})\tau}\nu_{i'}(\tau),
\end{equation} 
\begin{equation}
\nu_i(t) \simeq  \frac{1}{2} \int_0^t d\tau
\sum_{i'}(e_i v e_{i'})
e^{-i(\omega_i + \omega_{i'} -\omega_l)\tau}\mu_{i'}(\tau)
\end{equation}
(fast oscillating terms 
$\sim \exp{[\pm i(\omega_i + \omega_l \pm \omega_{i'})t]}$ and
$\sim \exp{[\pm i(\omega_{i'} + \omega_l \pm \omega_{i})t]}$ are neglected).
One sees that the equation for $\nu_i$ contains the sum $D(\tau) 
= \sum_{i'}\bar{e}_{i'} e^{-i\omega_{i'}\tau}\mu_{i'}(\tau)$:
\begin{equation}
\nu_i(t)=\frac{i}{2} \int_0^t d\tau (\bar{e}_i v D(\tau))
e^{i(\omega_l - \omega_i)\tau}
\end{equation}
Inserting (11) to (10) and then (10) to $D(t)$ one finds the following 
integral equation 

\begin{equation}
D(t)= D_0(t)+ \int_0^t d\tau \int_0^{\tau} d\tau^{'} G(t-\tau)vG(\tau-\tau^{'})
v e^{i\omega_l(\tau^{'}-\tau)} D(\tau^{'})
\end{equation}
where $D_0(t)=\sum_i \bar{e}_ie^{-i\omega_i t +i\phi_i}$,
$$
G_{nn'}(t) = \frac{i}{2}\sum_i e^{-i\omega_i t} \bar{e}_{in} \bar{e}_{in'}
$$
is retarded Green's function of phonons.
This equation can be easily solved by means of half-axis Fourier transform
(the order of integrations should be changed; then upper and lower limits of 
integrals over $\tau'$, $\tau-\tau'$ and $t-\tau$ will be $\infty$ and $0$).
One gets \cite{hizhrev}
\begin{equation}
D(\omega) = [I + G(\omega)vG^*(\omega_l-\omega)v]^{-1} D_0(\omega),
\end{equation}
where the functions of $\omega$ are the half-axis Fourier transforms of the
corresponding time-dependent functions.
This allows one to find also $\nu_i$ and $\mu_i$.

Given above value  of $g_i(t)$ at large $t$
leads to the following asymptotical relation:
$$
\hat{q}_{i}(t) \simeq \sqrt{\hbar/2\omega_{i}}
[e^{-i\omega_it}(\mu_i \hat{a}_{i} + \nu_i^*\hat{a}_{i}^{+})
+ e^{i\omega_it}(\mu_i^* \hat{a}_i^+ + \nu_i\hat{a}_i)],\,\,\,t\gg 1/\omega_i
$$
The positive frequency time dependence $\sim e^{-i\omega_i t}$ is 
characteristic for the destruction operator. Consequently,
phonon operators transform in time according to
$$
\hat{b}_i \simeq \mu_i\hat{a}_i + \nu_i^* \hat{a}_i^+.
$$
As a result the number of phonons in time $t \gg \omega_l^{-1}$ is not equal 
zero:
$\langle 0\vert\hat{b}_i(t)b_i(t)\vert 0\rangle=
\langle \vert\nu_i(t)\vert ^2 \rangle$ 
(here $\vert 0\rangle$ is the initial zero-point state: 
$\hat{a}_i\vert 0\rangle=0$, $\langle...\rangle$ denotes the averaging over 
$\phi_i$), i.e. phonons are created in the lattice. The rate of the energy 
generation equals
\begin{eqnarray}
I(t) \simeq \sum_i 
\hbar\omega_i (2n_i +1) d\langle \vert\nu_i(t)\vert ^2 \rangle/dt 
\end{eqnarray}
($t\gg\omega_i^{-1}$; $n_i$ is the number of initial phonons). 
%RELAXATION JUMPS
\subsection{Relaxation jumps}
By applying the energy conservation law one finds the rate of energy loss by 
the local mode: $-\dot{E}_l=I(t)$. Inserting expressions (12) and (14) to (15) 
one obtains \cite{hizhrev}
\begin{equation}
\frac{dE_l(t)}{dt} \simeq -\frac{\hbar \omega_l}{2\pi} \int_0^{\infty}
d\omega \mbox{Sp} (P(\omega_l -\omega )P(\omega))(1+2n(\omega ))
\end{equation}
where $n(\omega)=(e^{\hbar\omega/kT}-1)^{-1}$,
\begin{equation}
P(\omega ) =  \lbrace I-G(\omega)vG^*(\omega_l-\omega )v\rbrace ^{-1}
ImG(\omega);
\end{equation}
$v \sim A_l$ as well as $\omega_l$ and $G(\omega)$ also slowly depend on $t$.

Matrix under $Sp$ in (16) may be diagonalized. This is achieved by 
diagonalizing the even matrix  $\sqrt{v_n v_{n'}} G_{nn'}(\omega)$.
In this representation  $I=\sum_m I_m$,
\begin{equation}
I_m = \frac{\pi \omega_l v_m^2}{8}\int_{\omega_l-\omega_m}^{\omega_m} d\omega
\frac{\rho_m(\omega)\rho_m(\omega_l - \omega)(1+2n(\omega))}
{\vert 1- v_m^2 \tilde{G}_{2m}(\omega_l;\omega)\vert^2},
\end{equation}
where $\rho_m (\omega) = 2(\pi)^{-1} Im \bar{G}_m(\omega)$ is the projected
one-phonon density function,
\begin{equation}
\bar{G}_m (\omega) =
 v_m^{-1}\sum_{nn'}\sqrt{v_n v_{n'}} G_{nn'}(\omega) S_{nm}S_{n'm},
\end{equation}
$v_m= (2/\pi)\int_0^{\omega_m}
Im\sum_{nn'}\sqrt{v_n v_{n'}} G_{nn'}(\omega) S_{nm}S_{n'm} d\omega$,
$\tilde{G}_{2m}(\omega_l;\omega)= \bar{G}_m(\omega)\bar{G}_m(\omega-\omega_l)$
is the two-phonon Green's function.

Neglecting in (16) (and in (18)) higher than second-order terms with respect to
$v$, one finds $\dot{E}_l \sim A^2 \sim E_l$ and the exponential decay law
of local modes associate with lattice defects; the value of 
$\gamma = \dot{E}_l/E_l$ is in accordance with the one given by the 
perturbation theory \cite{klemens}. This approximation holds in the case of 
small $E_l$ (remind that ILMs with small $E_l$ do not exist; therefore their 
final stage of relaxation is different; see below). In the opposite case of 
very large $E_l$ the relaxation rate decreases with increasing of $E_l$. This 
result is opposite to that given by the perturbation theory; it means that at 
large amplitude relaxation, being initially slow, accelerates in time. This 
continues  till to the energy $E_{cr}$ when the resolvents under integral in 
(16) (and in (18)) turn to zero (for $\omega = \omega_l/2$ and in some cases 
for other $\omega$, for which $Im \bar{G} = 0)$. Near such energy 
$-\dot{E}_l \sim |E_l-E_{cr}|^{-1} \sim 1/sqrt{|t-t_{cr}|}$, i.e. sharp 
relaxation jump takes place accompanied by generation of burst of phonons.
After first jump several other jumps  may occur (see below).
%ODD MODE
\section{Odd local mode in perfect and perturbed cubic lattice}
Formula (16) give a transcendental equation for $E_{l}(t)$, which can be step 
by step evaluated in time, starting with small values of $t$. For every time
moment $t$ one should first calculate the rate of the relaxation of the
energy of the local mode $E_{l}(t)$ which is equal to $I(t)$, then find 
$E_{l}$ for the next $t$ and afterwards, solving classical equations of
motion (4) and (5), find all other parameters for the next $t$ and proceed
further until the relaxation process reaches the end.
To perform calculations of the relaxation law ) one has to find first matrices 
$G(\omega)$ and $v$. The Green's function matrix$G(\omega)$ 
can be found by applying Lifshits formula
\begin{equation}
G_{nn'} = G_{nn'}^{(0)} + \sum_{n_1} G^{(0)}_{nn'}V_{nn_1}G_{n_{1} n}
\end{equation}
where $V_{nn'}$ is the perturbation matrix ($V$ contains $V^{(2)}$),
$G_{nn'}^{(0)}$
are the Green's functions of the perfect lattice. 
These functions can be calculated by standard methods of
lattice dynamics. The matrix $v$ is given by (6) and it can be directly
evaluated from anharmonic potentials after solving classical equations of
local dynamics for given $E_l$. 

We present here calculations of odd local modes in a simple cubic lattice. 
Within the approximation of the nearest neighbors interaction the potential 
operator has the form
\begin{equation}
\hat{V} = \sum_{\alpha}\sum_{\vec{n}}\sum_{m=1}^{\infty} \frac{1}{m} V^{(m)}_{
\vec{n}_{\alpha}} {\big (} \hat{R}_{\vec{n}_{\alpha}} {\big )}^{m}\,,
\end{equation}
where $\alpha = x,\:y,\:z$ are the directions of crystal axes, 
$\vec{n} = (n_{x},n_{y},n_{z})$ is the vector of the lattice sites, 
$\vec{n}_{\alpha}$ is the vector of the site nearest to $\vec{n}$ in $\alpha$ 
direction, $V^{(m)}_{\vec{n}\alpha} = V_m$ is the $m$-th derivative of the 
pair potential between atoms (ions) $\vec{n}$ and $\vec{n}_{\alpha}$ at their
distance $R_{0\vec{n}_{\alpha}}$
\begin{equation}
\hat{R}_{\vec{n}_{\alpha}} = \sqrt{(R_{0 n_{\alpha}} + \hat{r}_{\alpha 
\vec{n}_{\alpha}})^{2} + \hat{r}^{2}_{\vec{n}_{\alpha}}-
\hat{r}_{\alpha \vec{n}_{\alpha}}^{2}} - R_{0\vec{n}\alpha}
\end{equation}
is the operator of distance between the nearest neighbors in the $\alpha$-
direction, $\hat{r}_{\beta\vec{n}_\alpha}=q_{\beta\vec{n}}-q_{\beta 
\vec{n}_{\alpha}}$, $q_{\beta}$ is the $\beta$-component of the displacement 
vector $\vec{q}_{\vec{n}}$ of the atom $\vec{n}$, 
$\hat{r}_{\vec{n}_{\alpha}}^{2} = \hat{r}^{2}_{x\vec{n}_{\alpha}} +
\hat{r}^{2}_{y\vec{n}_{\alpha}} + \hat{r}^{2}_{z \vec{n}_{\alpha}}$.
By expanding $\hat{V}$ in the power series of displacement operators 
$\hat{r}_{\alpha\vec{n}_{\alpha}}$one gets
\begin{eqnarray}
\hat{V} &=& \frac{1}{2}\sum_{\alpha , \vec{n}_{\alpha}}
[V_2\hat{r}^2_{\alpha \vec{n}_{\alpha}} + 
V'_2(\hat{r}^2_{\vec{n}_{\alpha}}-\hat{r}^2_{\alpha \vec{n}_{\alpha}}) +
\nonumber\\
&&\frac{1}{3}V_3\hat{r}^3_{\alpha n_{\alpha}} + 
V'_3 \hat{r}_{\alpha n_{\alpha}}(\hat{r}^2_{\vec{n}\alpha}-
\hat{r}^2_{\alpha\vec{n}_{\alpha}})+
\frac{1}{12} V_4\hat{r}^4_{\alpha n_{\alpha}}  \nonumber \\
&&\frac{1}{2}V'_4 \hat{r}^2_{\alpha n_{\alpha}} (\hat{r}^2_{n_{\alpha}}-
\hat{r}^2_{\alpha n_{\alpha}}) +
\frac{1}{4} V'' (\hat{r}^2_{n_\alpha}-\hat{r}^2_{\alpha n_{\alpha}})^2 +...],
\end{eqnarray}
where
\begin{eqnarray}
V'_2 &=& V_1 R_0^{-1},\quad V'_3 = (V_2 - V'_2)R_0^{-1}, \nonumber \\
V'_4 &=& R_0^{-1}V_3 - 2R_0^{-2}(V_2 - V'_2),\quad
V''_4 = R_0^{-2} (V_2 - V'_2),
\end{eqnarray}
$V_{2}$, $V_{3}$ and $V_{4}$ make account of the central, while $V'_{2}$, 
$V'_{3}$, $V'_{4}$ and $V''_{4}$ of the noncentral forces. The potential 
considered does not take account of the covalent interaction which leads to 
the chemical bonding. This (covalent) interaction can, however, be easily 
included in calculations by introducing additional terms of the type $V_{2}$, 
$V_{3}$ and $V_{4}$.

%CENTRAL FORCES
\subsection{Central forces}
As it is known
harmonic noncentral springs $V'_2$ are normally $5$ to $10$ times smaller 
than the central springs $V_2$. Our calculations of $V_3$ and $V'_3$ for
alkali halides show that $V'_3/V_3$ is even smaller than $V'_2/V_2$
($V'_3$ is twenty-thirty times smaller than $V_3$).

The same holds also for
quartic and higher order anharmonic terms: the higher order anharmonicity the
smaller are corresponding noncentral interactions as compared to central ones.
Therefore, as a first step, only central forces may be accounted.
We calculate a strong local  vibration of light 
impurity or host atom (ion) situated at the origin of our reference frame. In
this case solutions of classical equations of motion, 
corresponding to the local mode, satisfy the conditions:
$|A_{0}| \gg |A_{\vec{n}}|$. This allows one to suppose that the mode is well 
localized on the atom at the site $n=0$. Then in the approximation of central 
forces only coordinates of the central atom and of the directed to 
this atom components of coordinates of the nearest neighbor atoms contribute 
to the relaxation. We chose:
$\hat{q}_1 = \hat{x}_0$,
$\hat{q}_2 = \hat{x}_{1x}$,
$\hat{q}_3 = \hat{x}_{-1x}$,
$\hat{q}_4 = \hat{y}_0$,
$\hat{q}_5 = \hat{y}_{1y}$,
$\hat{q}_6 = \hat{y}_{-1y}$,
$\hat{q}_7 = \hat{z}_0$,
$\hat{q}_8 = \hat{z}_{1z}$,
$\hat{q}_9 = \hat{z}_{-1z}$,
In this representation the impurity induces change of matrix 
$\bar{V}_2$; corresponding perturbation matrix is
\[ V = \left (\begin{array}{ccc}
w_x&0&0\\\noalign{\medskip}
0&w_y&0\\\noalign{\medskip}
0&0&w_z\end{array}
\right ) , \;\;
w_{\alpha} = \left (\begin{array}{ccc}
\beta_{\alpha}&-\gamma_{\alpha}&-\gamma_{\alpha}\\\noalign{\medskip}
-\gamma_{\alpha}&\gamma_{\alpha}&0\\\noalign{\medskip}
-\gamma_{\alpha}&0&\gamma_{\alpha}\end{array}
\right ) , \]
($\alpha = x,y,z$). In harmonic approximation $\beta$ and $\gamma$ do
not depend on $\alpha$: $\gamma = \Delta V_2$ is the change of the magnitude 
of central elastic constants due to defect atom, 
$\beta = 2\gamma+\omega^2 (1-M/M_0)$, $\omega$ is the frequency of
the normal mode, $M/M_0$ is the ratio of impurity and host atom masses.
Quartic anharmonicity leads to amplitude dependent corrections of
elastic constants and to their dependence of $\alpha$:
\begin{equation}
\beta_{\alpha} = \omega^2(1- \frac{M}{M_0})+2\gamma_{\alpha},\quad
\gamma_{\alpha} = (\Delta V_2+\frac{1}{2}V_4 A^2_{\alpha}),
\end{equation}
 $A_{\alpha}$ is the $\alpha$'s Cartesian 
component of the amplitude of the local mode.

Perturbed Green's functions are
\[ G = \left (\begin{array}{ccc}
G_x&0&0\\ \noalign{\medskip}
0&G_y&0\\ \noalign{\medskip}
0&0&G_z\end{array}
\right ) , \;\;
G_{\alpha} = \left (\begin{array}{ccc}
G_{\alpha 11}&G_{\alpha 12}&G_{\alpha 12}\\ \noalign{\medskip}
G_{\alpha 12}&G_{\alpha 22}&0\\ \noalign{\medskip}
G_{\alpha 12}&0&G_{\alpha 22}\end{array}
\right ) , \]
where $G_{\alpha nn'} = ([I-G^{(0)}w_{\alpha}]^{-1}G^{(0)})_{nn'}$,
\begin{equation}
G^{(0)}_{nn} = 
\sum_{\nu \vec{k}}\frac{e_{n \nu \vec{k}}^2}
{\omega^2-\omega^2_{\nu\vec{k}}}, \\\;\;\;
G^{(0)}_{12} =
\sum_{\nu \vec{k}}\frac{e_{1\nu \vec{k}} e_{2\nu \vec{k}}\cos{(k_{x}d)}}
{\omega^2-\omega^2_{\nu\vec{k}}},
\end{equation}
are Green's functions of perfect lattice, 
$d$ is the lattice constant, $\vec{k}$ is the wave vector 
of phonon, $\nu$ is the phonon branch, $\omega_{\nu\vec{k}}$ is the
frequency of phonon, $e_{n \nu\vec{k}}$ is the projection of the
polarization vector of phonon onto $x$-component of the atom. 
We use for the Green's functions $G^{(0)}$ their values
calculated within shell model \cite{shell} (see Fig.  ).

%FIG. GREENsFUNCTIONS
Energy of the local mode depends on the anharmonicity parameter $V_4$:\
\begin{equation}
E_l \simeq \frac{1}{2}M\omega_l^2 A_l^2 + \frac{1}{64}V_4(A_x^4+A_y^4+A_z^4).
\end{equation}
The same holds for the frequency of the mode, which changes with the amplitude.

In a perfect lattice the perturbation matrix $V \sim V_4 A_l^2$. 
In the case of light central atom this perturbation of local dynamics leads to
appearance of almost fully localized mode (ILM) if the corresponding atom is 
light, quartic anharmonicity is hard and amplitude of the mode is large 
(see Fig.  ).

%FIG. FREQUENCIES
In the case under consideration time dependent interaction Hamiltonian equals
$$
W = 2 \cos{\omega_l t} \sum_{n,n'=1}^{9}v_{nn'}\hat{q}_n \hat{q}_{n'},
$$
where the matrix $v$ is determined by the coefficients of the quadratic
operators of the cubic anharmonicity term with
$r_{\alpha \vec{n}_{\alpha}} = 
A_{\alpha} \cos{\omega_l t} + q_m - q_{m+n}$
(here $m=1$ if $\alpha = x$, $m=2$ if $\alpha=y$ and $m=3$ if $\alpha=z$; 
$n=1,2$). This matrix equals
\[ v = \left (\begin{array}{ccc}
v_x&0&0\\\noalign{\medskip}
0&v_y&0\\\noalign{\medskip}
0&0&v_z\end{array}
\right ) , \;\;
v_{\alpha} = \frac{A_{\alpha}V_3}{2\sqrt{MM_1}}
\left (\begin{array}{ccc}
0&-1&1\\\noalign{\medskip}
-1&\varsigma&0\\\noalign{\medskip}
1&0&-\varsigma\end{array}
\right ) , \]
where $\varsigma = \sqrt{M/M_1}$.

We performed calculation of the local dynamics associated with the strong 
local vibrations of $F^-$ and $Na^-$ ions and their anharmonic relaxation in  
different alkali halide crystals (see Figs.  ). 
%FIGs. CENTRAL RELAXATION
Obtained results show existence of very sharp relaxation jumps at amplitudes 
$\gtrsim 0.3 \AA$. Although the total number of the jumps is rather large 
(this number depends on initial 
amplitude and on direction of vibration as well as on the system considered),
on the whole the main part of energy 
of the vibration is commonly released in rather short time of the order of few 
picoseconds, so that emission of phonons resembles an sharp explosion. 
In the case of the local mode associated with impure atom
a part of energy is lost in exponential (final) stage of the 
relaxation, while the ILMs do not have such a stage: they life-time is finite.
Rather remarkable is also slow-down of relaxation in initial
stage causing delay of the emission of phonon bursts. Depending on initial 
amplitude this delay may reach hundreds of picoseconds ore more.


%NONCENTRAL FORCES
\subsection {Noncentral forces}

Although noncentral anharmonic interactions are remarkably weaker than central 
ones their number is larger. Therefore it is of interest to account them.
These interactions  switch-on all 21 Cartesian coordinates of the central 
atom and its 6 nearest neighbors to the relaxation. It is convenient to chose 
these coordinates as follows (see Fig.~1):
$\hat{q}_{1} = \hat{x}_{0}$, 
$\hat{q}_{2} = \hat{y}_{0}$, 
$\hat{q}_{3} = \hat{z}_{0}$,
$\hat{q}_{4} = \hat{x}_{1_{x}}$, 
$\hat{q}_{5} = \hat{y}_{1_{x}}$,
$\hat{q}_{6} = \hat{z}_{1_{x}}$, 
$\hat{q}_{7} = \hat{x}_{-1_{x}}$, 
$\hat{q}_{8} = \hat{y}_{-1_{x}}$, 
$\hat{q}_{9} = \hat{z}_{-1_{x}}$,
$\hat{q}_{10} = \hat{1}_{1_{y}}$, 
$\hat{q}_{11} = \hat{x}_{1_{x}}$, 
$\hat{q}_{12} = \hat{z}_{1_{y}}$,
$\hat{q}_{13} = \hat{y}_{-1_{y}}$, 
$\hat{q}_{14} = \hat{x}_{-1_{y}}$,
$\hat{q}_{15} = \hat{z}_{-1_{y}}$, 
$\hat{q}_{16} = \hat{z}_{1_{z}}$, 
$\hat{q}_{17} = \hat{x}_{1_{z}}$, 
$\hat{q}_{18} = \hat{y}_{1_{z}}$, 
$\hat{q}_{19} = \hat{z}_{-1_{z}}$, 
$\hat{q}_{20} = \hat{x}_{-1_{z}}$, 
$\hat{q}_{21} = \hat{y}_{-1_{z}}$;
$A = \sqrt{A_{x}^{2}+A_{y}^{2}+A_{z}^{2}}$.
In this representation 
the perturbation matrix of lattice dynamics equals 
\[ V = \left (\begin{array}{cccc}
\nu_0&-\bar{\nu}_1&-\bar{\nu}_2&-\bar{\nu}_3\\\noalign{\medskip}
-\bar{\nu}_1^{\top}&\tilde{\nu}_1&0&0\\\noalign{\medskip}
-\bar{\nu}_2^{\top}&0&\tilde{\nu}_{1'}&0\\\noalign{\medskip}
-\bar{\nu}_3^{\top}&0&0&\tilde{\nu}_{1''}\end{array}
\right ) , \]
where $\nu_0 = \tilde{\beta} I_3$ is $3$ x $3$ matrix,  
$\bar{\nu}_{i^{(','')}} =(1,1)$ x $\nu_{i^{(','')}}$ are $3$ x $6$ matrixes,
$\tilde{\nu}_1 = I_2$ x $\nu_1$ are $6$ x $6$ matrixes,
$I_n$ is $n$x$n$-unit matrix,
$\gamma$ and 
$\gamma'$ are changes of central and non-central elastic springs due to the 
impure central ion, $\tilde{\beta} = \omega^2 (1-M/M_0) + 2\gamma + 4\gamma'$, 
\[ \nu_1 = \left (\begin{array}{ccc}
\gamma&0&0\\\noalign{\medskip}0&\gamma'&0\\\noalign{\medskip}0
&0&\gamma'\end {array}\right ),\;\;
\nu_2 = \left (\begin{array}{ccc}
0&\gamma'&0\\\noalign{\medskip}\gamma&0&0\\\noalign{\medskip}0
&0&\gamma'\end{array}\right ) ,\;\;
\nu_3 = \left (\begin{array}{ccc}
0&\gamma'&0\\\noalign{\medskip}0&0&\gamma'\\\noalign{\medskip}\gamma
&0&0\end{array}\right ) .\]
In harmonic approximation $\nu_1 = \nu_{1'} = \nu_{i''}$.
Quartic anharmonicity causes amplitude-dependent renormalization
of the elastic springs and leads to the following corrections of
(additions to) given above matrixes $\nu$:  

1) (100)-direction: $\nu'_{1'} = \nu'_{1''} = \nu'_2$, 
\[ \nu'_0 = \left (\begin{array}{ccc}
\zeta_1&0&0\\\noalign{\medskip}0&\zeta_2&0\\\noalign{\medskip}0
&0&\zeta_2\end {array}\right ),\;\;
\nu'_1 = \left (\begin{array}{ccc}
\delta_1&0&0\\\noalign{\medskip}0&\delta_2&0\\\noalign{\medskip}0
&0&\delta_2\end {array}\right ),\;
\nu'_2 = \left (\begin{array}{ccc}
0&0&0\\\noalign{\medskip}0&0&0\\\noalign{\medskip}0
&0&\delta_2\end{array}\right ) ,\;\;
\nu'_3 = \left (\begin{array}{ccc}
0&0&0\\\noalign{\medskip}0&0&\delta_2\\\noalign{\medskip}0
&0&0\end{array}\right ) ,\]
where $\delta_1 = A^2(V_4 + 6V'_4)/4,
\delta_2 = A^2V'_4/4, \zeta_1 = 2\delta_1, \zeta_2 = 4\delta_2$; 
energy correction equals $E'_l = A^4(V_4 + 6V'_4)/64$;  

1) (110)-direction: $\nu'_{1'} = \nu'_1$, 
\[ \nu'_0 = \left (\begin{array}{ccc}
\xi_1&\bar{\sigma}&0\\\noalign{\medskip}\bar{\sigma}
&\xi_1&0\\\noalign{\medskip}0
&0&\xi_2\end {array}\right ),\;
\nu'_1 = \left (\begin{array}{ccc}
\sigma_1&\sigma_2&0\\\noalign{\medskip}\sigma_2&\sigma_2&0\\\noalign{\medskip}0
&0&\sigma_2\end {array}\right ),\;
\nu'_2 = \left (\begin{array}{ccc}
\sigma_2&\sigma_2&0\\\noalign{\medskip}0&0&0\\\noalign{\medskip}\sigma_1
&\sigma_2&\sigma_2\end{array}\right ),\;
\nu'_3 = \left (\begin{array}{ccc}
0&0&0\\\noalign{\medskip}\sigma_2&0&0\\\noalign{\medskip}0
&0&0\end{array}\right ),\;
\nu'_{1''} = \left (\begin{array}{ccc}
\sigma_2&0&0\\\noalign{\medskip}0&0&0\\\noalign{\medskip}0
&0&0\end{array}\right ),\] 
where $\sigma_1 = A^2(V_4 + 7V'_4)/8$,
$\sigma_2 = A^2V'_4/8$,  $\bar{\sigma} = 4\sigma_2$, 
$\xi_1 = 2\sigma_1 + 6\sigma_2$, $\xi_2 = 6\sigma_2$; 
energy correction equals $E'_l = A^4(V_4 + 12V'_4)/128$;  

1) (111)-direction: $\nu'_{1'} = \nu'_{1''} = \nu'_1$, 
\[ \nu'_0 = \left (\begin{array}{ccc}
\phi&f&f\\\noalign{\medskip}f&\phi&f\\\noalign{\medskip}f
&f&\phi\end {array}\right ),\;\;
\nu'_1 = \left (\begin{array}{ccc}
\kappa_1&\kappa_2&\kappa_2\\\noalign{\medskip}\kappa_2&\kappa_3&0\\\noalign
{\medskip}\kappa_2
&0&\kappa_3\end {array}\right ),\;\;
\nu'_2 = \left (\begin{array}{ccc}
\kappa_2&\kappa_3&0\\\noalign{\medskip}\kappa_1&\kappa_2&\kappa_2\\\noalign
{\medskip}\kappa_2
&0&\kappa_3\end{array}\right ) ,\;\;
\nu'_3 = \left (\begin{array}{ccc}
\kappa_2&\kappa_3&0\\\noalign{\medskip}\kappa_2&\kappa_3&\delta_2\\\noalign
{\medskip}\kappa_1
&\kappa_2&\kappa_2\end{array}\right ) ,\]
where $\kappa_1 = A^2(V_4 + 8V'_4)/12$,
$\kappa_2 = A^2V'_4/6$, $\kappa_3 = \kappa_2/2$, $\phi=2\kappa_1 + 4\kappa_2$,
$f = 8\kappa_3$; 
energy correction equals $E'_l = A^4(V_4 + 18V'_4)/192$.
Corrections 
$\sim V''_4$ are neglected; they are at least one order of magnitude smaller
than smallest accounted correction $\sim V'_4$.  
As one sees the main effect of the noncentral anharmonic interactions 
to the static local dynamics consists in renormalization of 
central elastic constants: $A^2V_4$ is 
replaced by $A^2(V_4 + 6V'_4)$. This allows one to improve the central-force 
approximation by replacing $V_4$ by $\bar{V}_4 = V_4 + 6V'_4$.

Green's function matrix of a perfect lattice in the basis of 21 coordinates
equals
\[ G^{(0)} = \left (\begin{array}{cccc}
J_0&\bar{m}_1&\bar{m}'_1&\bar{m}''_1\\\noalign{\medskip}
\bar{m}_1^{\top}&\tilde{J}&\tilde{m}_2&\tilde{m}'_2\\\noalign{\medskip}
\bar{m}_{1}^{\prime\top}&\tilde{m}_2&\tilde{J}&\tilde{m}''_2\\
\noalign{\medskip}
\bar{m}_1^{\prime\prime\top}&\tilde{m}_2^{\prime\top}&\tilde{m}''_2&\tilde{J}
\end{array}
\right ) , \]
where $\bar{m} = (1,1)$ x $m$ are 3x6-matrixes, while $\tilde{J}$ and
$\tilde{m}$ are 6x6-matrixes:
\[ \tilde{J} = \left (\begin{array}{cc}
J&J_1\\\noalign{\medskip}
J_1&J\end{array}\right ) , \;\;
\tilde{m} = \left (\begin{array}{cc}
m&m\\\noalign{\medskip}m&m\end{array}\right ) ,\]
$J_0 = G^{(0)}_{1,1}$ x $I_3$, $J = G^{(0)} _{4,4}$ x $I_3$, 
\[ J_1 = \left (\begin{array}{ccc}
G^{(0)}_{4,7}&0&0\\\noalign{\medskip}0&G^{(0)}_{5,8}&0\\\noalign{\medskip}0
&0&G^{(0)}_{5,8}\end {array}\right ),\;\;
m_1 = \left (\begin{array}{ccc}
G^{(0)}_{1,4}&0&0\\\noalign{\medskip}0&G^{(0)}_{1,11}&0\\\noalign{\medskip}0
&0&G^{(0)}_{1,11}\end {array}\right ),\;
m_2 = \left (\begin{array}{ccc}
G^{(0)}_{4,10}&G^{(0)}_{4,11}&0\\\noalign{\medskip}
G^{(0)}_{4,11}&G^{(0)}_{4,10}&0\\\noalign{\medskip}
o&o&G^{(0)}_{6,12}\end{array}\right ) .\]
The matrix $m'_1$ can be obtained from $m_1$ by permutation of the first and
the second lines, while $m''_1$ can be obtained from $m'_1$ by permutation of 
the second and the third lines; analogously the matrix $m'_2$ can be obtained
from $m_2$ by permutation of the second and the third lines, while $m''_2$ can
be obtained from $m'_2$ by permutation of the second and the third columns.
One sees that altogether 9 different Green's functions enter the matrix 
$G^{(0)}$; these functions as well as the polarization vectors entering these 
functions are depicted below: 
$G^{(0)}_{1,1} \sim e_{0x}^2$, 
$G^{(0)}_{1,4} \sim e_{0x}e_{1x}\cos{k_xd}$, 
$G^{(0)}_{1,11} \sim e_{0x}e_{1x}\cos{k_yd}$, 
$G^{(0)}_{4,4} \sim e_{1x}^2$, 
$G^{(0)}_{4,7} \sim e_{1x}^2\cos{2k_xd}$, 
$G^{(0)}_{4,10} \sim e_{1x} e_{1y}\cos{k_xd}\cos{k_yd}$,
$G^{(0)}_{4,11} \sim e_{1x}^2\cos{k_xd}\cos{k_yd}$,
$G^{(0)}_{5,8} \sim e_{1x}^2\cos{2k_yd}$, 
$G^{(0)}_{6,12} \sim e_{1x}^2\cos{k_yd}\cos{k_zd}$ 
(here $e_{0\alpha}$ correspond to the central ion, while $e_{1\alpha}$ 
correspond to the nearest neighboring ions).

The time-dependent part of the phonon Hamiltonian with account
of the noncentral cubic anharmonicity equals
\begin{equation}
W = 2\cos{\omega_{l} t}
\sum_{n,n'=1}^{21}v_{nn'}\hat{q}_{n}\hat{q}_{n'}\,.
\end{equation}
Here the relaxation matrix $v$ equals
\[ v =  \frac{A_lV'_3}{2\sqrt{lMM_1}}
\left (\begin{array}{cccc}
0&-\bar{\mu}_1&-\bar{\mu}_2&-\bar{\mu}_3\\\noalign{\medskip}
-\bar{\mu}_1^{\top}&\tilde{\mu}_1&0&0\\\noalign{\medskip}
-\bar{\mu}_2^{\top}&0&\tilde{\mu}_4&0\\\noalign{\medskip}
-\bar{\mu}_3^{\top}&0&0&\tilde{\mu}'_4\end{array}
\right ) , \]
where $l = 1$ for the local vibration in(100) direction, $l = 2$ for the (110)
direction, and $l = 3$ for the (111) direction,
$\bar{\mu}_i =(-1,1)$ x $\mu_i$ are 3x6-matrixes,
while $\tilde{\mu}_i = \sqrt{M/M_1}\sigma_z  $x$  \mu_i$ is 
6x6-matrixes with $\sigma_z$
being third Pauli matrix;  $\mu$-matrixes for three directions of local
vibration are given below (superscripts $(1)$, $(2)$ and $(3)$ correspond to
(100), (110) and (111) directions of local vibration)

\[ \mu_1^{(1)} = \left (\begin{array}{ccc}
\kappa&0&0\\\noalign{\medskip}0&1&0\\\noalign{
\medskip}0&0&1\end{array}
\right ) ,\;\;
\mu_2^{(1)} = \left (\begin{array}{ccc}
1&0&0\\\noalign{\medskip}0&1&0\\\noalign{\medskip}0
&0&0\end{array}\right ), \;\;
\mu_3^{(1)} = \left (\begin{array}{ccc}
1&0&0\\\noalign{\medskip}0&0&0\\\noalign{\medskip}0
&1&0\end {array}\right ),\;\;
\mu_4^{(1)} = \left (\begin{array}{ccc}
0&1&0\\\noalign{\medskip}1&0&0\\\noalign{\medskip}0
&0&0\end{array}\right ) ,\]

\[ \mu_1^{(2)} = \left (\begin{array}{ccc}
\kappa&1&0\\\noalign{\medskip}1&1&0\\\noalign{
\medskip}0&0&1\end{array}
\right ),\;\;
\mu_2^{(2)} = \left (\begin{array}{ccc}
1&1&0\\\noalign{\medskip}\kappa&1&0\\\noalign{
\medskip}0&0&1\end{array}
\right ),\;\;
\mu_3^{(2)} = \left (\begin{array}{ccc}
1&0&0\\\noalign{\medskip}1&0&0\\\noalign{\medskip}
0&1&1\end {array}\right ),\;\;
\mu_4^{\prime (2)} = \left (\begin{array}{ccc}
0&1&1\\\noalign{\medskip}1&0&0\\\noalign{\medskip}
1&0&0\end{array}\right ), \]

\[ \mu_1^{(3)} = \left (\begin{array}{ccc}
\kappa&1&1\\\noalign{\medskip}1&1&0\\\noalign{
\medskip}1&0&1\end{array}
\right ),\;\;
\mu_2^{(3)} =  \left (\begin{array}{ccc}
1&1&0\\\noalign{\medskip}\kappa&1&1\\\noalign{\medskip}
1&0&1\end{array}\right ),\;\; 
\mu_3^{(3)} = \left (\begin{array}{ccc}
1&1&0\\\noalign{\medskip}1&0&1\\\noalign{\medskip}
\kappa&1&1\end{array}
\right ), \]
$\mu_4^{\prime (1)} = \mu_1^{(1)}$, $\mu_4^{(2)} = \mu_1^{(2)}$, 
$\mu_4^{(3)} = \mu_4^{\prime (3)} = \mu_1^{(3)}$.

The evolution of the energy of the local vibration 
starting value is the same for all three directions) is illustrated.
The corresponding rates of the decay are shown.
As it can be seen, the process of relaxation is determined by
the set of parameters for interaction matrix and by the initial energy of
the excitation of vibration. Each relaxation jump leads to the release 
of a large part of local
mode energy in several periods of vibration. Depending on the initial
conditions, the most stable
vibrations can be either in [100] or in [111] direction. This effect of
anisotropic relaxation is specific for the mechanism  described.

We should note that in the case of large $\kappa$, which corresponds to the
domination of covalent interaction between the central atom and its
nearest neighbor, the most stable is the local vibration in the
direction of the nearest neighbors.

\subsection{Temperature dependence}
We carried out a set of calculations for different temperatures. Generally,
we may conclude that for higher temperatures the damping of the local
mode is faster.

\section{Discussion}
As it was shown above, the relaxation of a strongly excited local mode is
highly nonexponential. The rate of the relaxation $\dot{E}_l$ is rather 
small at higher
values of the energy $E_{l}$; with time the relaxation proceeds and the 
energy decreases causing acceleration of the process;
There exist some critical values of the energy 
when the rate of relaxation reaches
extremely large values of the order of vibrational frequency. A large
amount of the energy of the local
vibration is released in a very short time by emission of burst of phonons
resulting in the jump-like decreasing of the energy of the mode. 
After first emission of the phonon burst relaxation rate drops fast 
and starts increasing again till to the next critical energy and so on.
The position and the number of the jumps depends on the crystal and on the 
impurity as well as on the amplitude of the vibrations. 
However, on the whole large part of energy of the local mode is 
released in rather short time of the order of few picoseconds.
In the case of a local mode associated with impurity the amount of energy
emitted in an explosion-like process may be small, if initial energy is also 
small. Unlikely to that the  decay of ILMs does not have the usual, 
exponential stage of relaxation at all and decay in an accelerating
process which resembles explosions of small black holes |cite{hawking}.  
and the emission of phonons by the mode resembles an sharp 
We demonstrate also that the process of relaxation is different for different 
directions of vibrations in the crystal lattice; depending on 
the initial conditions, vibrations in [100] or [111] directions appear to be 
more stable. These results, in our opinion, may be of importance for the 
understanding of the nature of defect formation by high-energy particles and 
by electronic excitations in solids, it can be also relevant for the 
description of chemical reactions. They also may open new ways in generation 
of ultrashort phonon  pulses. Finally, when the energy of the local mode takes
small values, the relaxation progresses with the exponential law.

In order to realize the physical meaning of such dramatic behavior we 
recall that every jump is associated with
a single configurational coordinate (and phonon band). 
Therefore, we discuss the reasons of a single relaxation peak 
associated with one configurational coordinate (and phonon band).

The fact, that relaxation slows down with the increasing of $E_{l}$ is
rather easy to understand: the limit  $E_{l} \rightarrow \infty$ (as well as 
the limit $M\rightarrow \infty$ for the case of the black hole)
corresponds to the problem when classical laws start working. 
Therefore, quantum emission of phonons  is suppressed in 
this case. One should not mix this effect
with the mechanism of the decreasing of the rate of relaxation of the
ILM described in \cite{ovchi}. The nature of the latter effect
is connected with the fact that enlargement of the amplitude of the 
ILM leads to enhancement of its frequency (due to the positive quartic
anharmonicity) and, consequently, to suppression of relaxation as a 
consequence of diminishing number of energetically allowed channels of the 
decay. The effect of slowing down of the relaxation rate, obtained in
present work, exists also in the case when the frequency does not
change with the variation of energy.

Thus, the nature of acceleration of relaxation at large energies is directly 
related to the decreasing of s.c. ``classicality''
of the strongly excited mode when relaxation proceeds. One can also 
understand why at some stage of relaxation this time dependence turns to the 
opposite one - to slow-down of the relaxation: the latter is characteristic 
for relaxation at small energies. 

We conclude that the nature of the relaxation peak at intermediate values of 
$E_{l}$ is rather general. It 
holds also for three-, four-, etc. phonon decay \cite{zhizh}. 

Finally we pay attention to the following features of the described 
process of generation of phonon bursts by strong local vibration:
\begin{enumerate}
\item A remarkable time delay between the excitation of the local vibration 
moment and the phonon burst, generated by this vibration; this delay may be 
of hundreds or thousands of vibrational
periods.
\item Quasimonochromatic spectrum of phonons, generated in the burst.
\item Dependence on the direction of vibration. Separate calculations
should be carried out for each type of the crystal lattice with a correct 
account of all possible contributions from vibrations of atoms.
\end{enumerate}
These effects may be used for an experimental observation of the process.

In conclusion we showed that quantum (and thermal) effects lead to  
the decay of a strong local vibration (intrinsic or associated with the 
impurity) with the time-dependence being strongly nonexponential. 
At very large energies the rate of decay is relatively low. It increases
when relaxation proceeds, at critical points it gets extremely high values.
Then the rate decreases and after that increases again until the energy 
reaches next critical point. The energy of the local mode drops in a short 
time at each critical point and a burst
of quasimonochromatic phonons is generated. The relaxation rate is higher for
higher temperatures, although the values of critical energies do not depend on
temperature. We also showed that the law of relaxation depends strongly on the
directions of vibration and its behavior is different for different
initial energies. Strong dependence of the relaxation on the directions
of the vibration obtained for cubic crystals is remarkable, since in the case
of the existence of chemical bonds the most stable are vibrations in the
direction of the bonds. This effect may have an important value for
chemical reactions and for the mechanism of defects formation in solids.
\section{Acknowledgment}
This research was supported by the Estonian Science Foundation, 
Grant No.~2274 and by the DAAD Grant. D.N. thanks NORDITA for hospitality
and financial support.

\begin{thebibliography}{99}
\bibitem{ovchi} 
A.A.Ovchinnikov, Zh. Eksp. Teor. Fiz. {\bf 57}, 263 (1969)
[Sov. Phys. JETP {\bf 30} 147 (1970)];
A.A.Ovchinnikov and N.S.Erihman, Usp. Fiz. Nauk {\bf 138}, 290 (1982)
[Sov. Phys. Usp. {\bf 25} 738 (1982)].
\bibitem{dolgov} 
A.S.Dolgov, Fiz. Tverd. Tela (Leningrad) {\bf 28}, 1641 (1986) 
[Sov. Phys. Solid State {\bf 28}, 907 (1986)]. 
\bibitem{sivtak} 
A.J.Sievers, S.Takeno, Phys. Rev. Lett. {\bf 61}, 970 (1988).
\bibitem{zavt} 
G.S.Zavt {\it et al.}, Phys. Rev. E {\bf 47}, 4108 (1993). 
\bibitem{page} 
J.B.Page, Phys. Rev. B {\bf 41}, 7835 (1990). 
\bibitem{kosevich}
A.M.Kosevich and A.S.Kovalev, Zh. Eksp. Teor. Fiz. {\bf 67}, 1793 (1974)
[Sov. Phys. JETP {\bf 40}, 891 (1974)].
\bibitem{procest} 
V.Hizhnyakov, {\it Proceedings of the  XIIth Symposium on the Jahn-Teller 
Effect} [Proc. Estonian Acad. Sci. Phys. Math. {\bf 44}, 364, (1995)]. 
\bibitem{hizhrev} 
V.Hizhnyakov, Phys.Rev. B {\bf 53}, 13981 (1996). 
\bibitem{hizhnev} 
V.Hizhnyakov, D.Nevedrov, Proc. Estonian Acad. Sci. Phys. Math. {\bf 44}, 
376 (1995);
V.Hizhnyakov and D.Nevedrov, Z. Phys. Chem. {\bf 201}, 301 (1997);
V.Hizhnyakov and D.Nevedrov, Pure and Appl. Chem. {\bf 69}, 1195 (1997).
\bibitem{hawking} 
S.W.Hawking, Nature (London),
{\bf 243}, 30 (1974); Commun. Math. Phys.  {\bf 43}, 199 (1975). 
\bibitem{birel} 
N.O.Birel and P.C.Davies, {\it Quantum Fields in Curved Space}
(Cambridge University Press, Cambridge, England, 1982).  
\bibitem{grib} 
A.A.Grib {\it et al} {\it Vacuum Quantum Effects in a Strong Field}, 
(Moscow, Energoizdat, 1988) (in Russian).  
\bibitem{unruh}
W.G.Unruh, Phys. Rev. D {\bf 14}, 870 (1976). 
\bibitem{hizhn}
V.Hizhnyakov, Quantum Optics {\bf 4}, 277 (1992); Proc. Estonian
Acad.  Sci. Phys. Math. {\bf 41}, 211 (1992); 
Symposium of the Foundations of Modern Physics (1994); 
K.V.Laurikainen (ed.), Editions Frontiers, p. 
199 (1994).  
\bibitem{marad} 
A.A.Maradudin {\it et al.} {\it Theory of Lattice Dynamics in 
Harmonic Approximation} (Academic, New York, 1963).  
\bibitem{maradu} 
A.A.Maradudin, {\it Theoretical and
Experimental Aspects of the Effects of Point Defects and Disorder of the 
Vibrations of Crystals} (Academic, New York 1966). 
\bibitem{zhizh}
V.Hizhnyakov, Z.Phys. B (1997)
\bibitem{chain}
V.Hizhnyakov and D.Nevedrov, Phys. Rev. B {\bf 56}, R2809 (1997).
\bibitem{hizh} 
V.Hizhnyakov, J. Phys. C: Solid State Physics {\bf 20}, 6073 (1987).  
\bibitem{klemens}
P.Klemens, Phys. Rev. {\bf 122}, 443 (1961).
\bibitem{heis}
W.Heisenberg, H.Euler, Z.Phys. {\bf 98}, 714 (1936).
\bibitem{econ}
E.N.Economou, {\it Green's Functions in Quantum Physics}
(Springer-Verlag, Berlin, 1983).
\bibitem{kristofel}
N.N.Kristofel, {\it Theoriya primesnyh centrov ,alyh radiusov
v ionnyh kristallah}  (Nauka, Moscow, 1974).
\bibitem{bilz}
H.Bilz, W.Kress, {\it Phonon Dispersion Relations in
Insulators} (Springer, Berlin, 1979).
\bibitem{shell} 
A.D.B.Woods, W.Cochran, and B.N.Brockhouse,
Phys. Rev. {\bf 119}, 980 (1960);
A.D.Woods, B.N.Brockhouse, R.A.Cowley, and W.Cochran,
Phys. Rev {\bf 131}, 1025 (1963).
R.A.Cowley, W.Cochran, B.N.Brockhouse, and A.D.B.Woods,
Phys. Rev {\bf 131}, 1030 (1963).
\bibitem{stone}
A.M.Stoneham, {\it Theory of Defects in Solids. Electronic Structure of Defects
in Insulators and Semiconductors} (Oxford University Press, London, England, 
1975).
\bibitem{multi}
V.Hizhnyakov, Z.Phys. B {\bf 104}, 675 (1997).
\bibitem{shell1}
B.G.Dick, in {\it Lattice Dynamics}, edited by R.F.Wallis, 
pp.159-187,(Pergamon, Oxford,
1965).
\bibitem{shell2}
A.D.Woods, W.Cochran, and B.N.Brockhouse,
Phys. Rev. {\bf 119}, 980 (1960).
\bibitem{shell3}
A.D.Woods, B.N.Brockhouse, and R.A.Cowley,
Phys. Rev. {\bf 131} 1025 (1963); R.A.Cowley, W.Cochran, B.N.Brockhouse, and
A.D.B.Woods, Phys. Rev. {\bf 131}, 1030 (1963);
G.Raunio and S.Rolandson, Phys. Rev. B {\bf 2}, 2098 (1970);
W.J.L.Buyers, Phys. Rev. {\bf 153}, 923 (1967).
\bibitem{note}
Obtained here formula (17) differs from the analogous formula (18)
in \cite{hizhrev}: formula (17) is obtained from general formula (13)
by an exact orthogonal transformation and is exact within the frames
of the theory; formula (18) in \cite{hizhrev}, however, is approximate one
while it
does not account nondiagonal elements of the Green's function matrix
$G_{nn'}$ which may differ from zero in the representation used 
in \cite{hizhrev}.
\end{thebibliography}

\newpage
\begin{center}
{\LARGE {\bf Figure captions}}
\end{center}

\bigskip

\noindent
{\bf Figure 1.} 21 configurational coordinates contributing to the
relaxation of the localised vibration. \\
{\bf Figure 2.} Frequency vs. displacement amplitude. $KCl:F$.\\
{\bf Figure 3.} Frequency vs. displacement amplitude. 
$NaI$.\\
{\bf Figure 4.} Rate of relaxation of the localized mode associated with
an impurity. $KCl:F$. $[100]$ -- solid line, $[110]$ -- long-dashed line,
$[111]$ -- dotted line.\\
{\bf Figure 5.} Displacement amplitude. 
$KCl:F$. $[100]$ -- solid line, $[110]$ -- long-dashed line,
$[111]$ -- dotted line.\\
{\bf Figure 6.} Rate of relaxation of the localized mode in a perfect
crystal. $NaI$. $[100]$ -- solid line, $[110]$ -- long-dashed line,
$[111]$ -- dotted line.\\
{\bf Figure 7.} Energy of the localized mode. 
$NaI$. $[100]$ -- solid line, $[110]$ -- long-dashed line,
$[111]$ -- dotted line.\\

\end{document}




