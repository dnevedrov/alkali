\documentclass{article}
\oddsidemargin=-0.5cm
\topmargin=0.0cm
\usepackage[dvips]{color,graphicx}
\textwidth=17cm
\usepackage[dvips]{color,graphicx}
\pagestyle{empty}
\textheight=23cm
\renewenvironment{thebibliography}[1]
       { \begin{list}{\arabic{enumi}.}
        {\usecounter{enumi}\setlength{\parsep}{0pt}
        \setlength{\itemsep}{0pt} \settowidth
        {\labelwidth}{#1.}
        \settowidth{\leftmargin}{#1.\rule{3mm}{0mm}}
        \sloppy}}{\end{list}}
\begin{document}
\begin{Huge}
\noindent
{\bf{
\textcolor{blue}
{We study hole states in the antiferromagnetically (AF) ordered CuO$_2$
planes of cuprate perovskites within the three-band
Hubbard model and with a self-consistent calculation of the Cu spin
polarization.
Both the Cu$-$O hybridization and the
O$-$O transfer are taken into account.
}


\vspace{10mm}

\noindent
\textcolor{magenta}{
The model Hamiltonian for CuO$_2$ planes in the
Hartree-Fock approximation of the original
Hubbard Hamiltonian reads
\textcolor{red}{\cite{oles,siebold}} }


$$ H=\sum_{\sigma}H_{0\sigma}+H_{\rm int} $$
%                                                 1
\[ H_{0\sigma}=\epsilon_d \sum_n n_{n\sigma}^d
          + \epsilon_p \sum_m n_{m\sigma}^p
       +T \sum_{nm} \left (d\,_{n\sigma}^\dagger p\,_{m\sigma}
                  + {\rm h.c.} \right ) \]
$$
  +t \sum_{mm'} \left (p\,_{m\sigma} ^\dagger p\,_{m'\sigma}
                  + {\rm h.c} \right )
$$
%                                                   2
\[ H_{\rm int} = U\sum_n n_{n\sigma}^d n_{n-\sigma}^d \]

\noindent
\textcolor{blue}
{$d$ $(d^\dagger)$ and $p$ $(p \, ^\dagger)$ are electronic annihilation
(creation) operators on Cu and O orbitals,
%\noindent
$U \approx 8$ eV, $T \approx 1$ eV, $t \approx 0.3$ eV,
$\epsilon = \epsilon_p - \epsilon_d \approx 3$ eV}
\textcolor{red}{\cite{schluter,stechel}}\textcolor{blue}{.}
%Hamiltonian (5) is nonlinear in wave functions. Therefore, in
%principle it can describe different spin-density distributions for the
%same set of parameters.

\vspace{10mm}
\noindent
In the AF ordered CuO$_2$ plane
the elementary cell is doubled
(the magnetic unit cell contains two CuO$_2$ units ).
The copper on-site energies are given by

\vspace{2mm}

\noindent
\textcolor{green}{
$\epsilon_{1 \sigma}=\epsilon_d + U \langle n_{1-\sigma}^d \rangle,$
$\epsilon_{2 \sigma}=\epsilon_d + U \langle n_{2-\sigma}^d \rangle.$ }

\vspace{10mm}
\noindent
\textcolor{blue}{
The Hamiltonian $ H_{\rm MF}^\sigma $ for the fixed
spin direction $\sigma = \,\uparrow $ in $k$-space  basis and
with the first and the forth states corresponding to the Cu states
$\vert d_1 \rangle $ and $\vert d_4 \rangle $
and the second, third, fifth, and sixth states corresponding to
the 4 oxygen $\vert p_1 \rangle , \vert p_2 \rangle , \vert p_3
\rangle $, and $ \vert p_4 \rangle$
states surrounding the first Cu-site (counted counter-clockwise
beginning from the right position), reads }


\newpage

\vspace{10mm}
\[
\langle \: k \sigma| H_{\rm MF}^\sigma | \, k \sigma\: \rangle =
\vspace{10mm}
\]
$$
=\left (
\begin{array}{cccccc}
\epsilon_1  & -2Ta_x & 2Ta_y & 0 & 2Ta_x^{\ast} & -2Ta_y^{\ast} \\
-2Ta_x^{\ast} & \epsilon_p & 2tc & 2Ta_x &  0  &       -2tb \\
2Ta_y^{\ast} & 2tc & \epsilon_p & -2Ta_y &   -2tb &       0 \\
 0  & 2Ta_x^{\ast} & -2Ta_y^{\ast} & \epsilon_2 & -2Ta_x & 2Ta_y \\
2Ta_x & 0 & -2tb & -2Ta_x^{\ast} &       \epsilon_p &       2tc \\
-2Ta_y & -2tb & 0 & 2Ta_y^{\ast} &       2tc &            \epsilon_p
\end{array}
\right )   \nonumber
$$


\vspace{2mm}
\textcolor{blue}{
$$
a_{x,y} =  e^{i k_{x,y} (a/2) }/2 ,
$$
%\vspace{0.02cm}
$$
b = \cos [(k_x + k_y)(a/2)] , \ \ \
%$$
%$$
c = \cos [(k_x - k_y)(a/2)],
$$
}
%                                   (11)
%                                   (12)
%                                   (13)
\vspace{0.1cm}
\noindent
\textcolor{green}{
$a$ is the lattice constant. }
%\vspace{10mm}


\vspace{1cm}



\noindent
{\bf The expectation values of polarization are obtained
from the self-consistent equations}
\textcolor{magenta}{
$$
\langle n_{\sigma}^d \rangle = \sum_k \mid \phi_{ik} \mid ^2 ,
$$
}
where $\phi_{ik}$ is the eigenvector of the Hamiltonian matrix,
corresponding to the eigenvalues $E_k$.


\noindent
The second band (empty in the undoped case) has 4 minima
in the ($\pm \pi / 2a ,\pm \pi /2a)$ points of the Brillouin zone.

\vspace{10mm}
\begin{center}
\textcolor{magenta}{
{\bf Wave packet of free hole.}}
\end{center}
\vspace{10mm}

\noindent
\textcolor{blue}{
We are interested in the behavior of
a large-size $L$ wave packet of an extra hole  added
to the AF ground state.
The wave-packet is presented in the form}

\vspace{10mm}
\noindent
$$
\vert \psi _L \rangle = \sum _{\vec{m} \prime} c_{\vec{m} \prime}
a^{+}_{\vec{m} \prime}
\vert 0 \rangle,
$$

\vspace{10mm}
\noindent
\textcolor{blue}{
where $\vert 0 \rangle $ is the
state with a filled lower Hubbard band.
We choose $c_{\vec{m} \prime}$ in the exponential form:}

\vspace{10mm}
\[
c_{\vec{m'}}= th (2 a'/L) e^{- 2 \left ( | m_{x'}| + |m_{y'}|
\right ) a' / L + i \pi m_{x'}}.
\]

\noindent
\textcolor{green}{$a'$ is a lattice constant in the AF case.}

\newpage
\noindent
\textcolor{blue}{
To find the value
of the barrier between free- and ferron-hole
states one should calculate the $L$-dependence of:}


\vspace{10mm}
\noindent
\textcolor{magenta}{
1) the energy of the wave packet of the free hole,}

\vspace{5mm}
\noindent
\textcolor{magenta}{
2) the energy of the latter packet when one spin is turned.}
\textcolor{blue}{These calculations are performed self-consistently.}

\vspace{10mm}
\noindent
One needs to calculate also the
self-energy correction $(E_{SE})$ caused by the hole-spin interaction.
At large $L$
$E_{SE}$  is approximately equal to the $1/N_0$-th part of
the self-energy correction of the $N_0$-particle state.

\noindent
The results  of calculations of $E_{SE}(L)$ as well as the full
energy of the wave packet
$\epsilon_L = E_{kin}(L) + E_{SE}(L)$
are presented in Fig.1.

\vspace{5mm}
\noindent

\newpage
\begin{center}
\textcolor{green}{
{\bf Effect of turned spin.}}
\end{center}

\noindent
To find the value of the barrier we
consider how the energy of the hole wave
packet of a large size is changed if one (central) Cu- spin
is turned. Two following contributions to the energy change should
be taken into account:

\noindent
\textcolor{magenta}{
1) The increase of the energy due to a change of magnetic interactions
with the turning of one spin.}
\textcolor{blue}{
We calculated   this energy by
means of the Lifshitz Green's function method.
For our parameters
$E^{(S)} \sim 0.15 eV$.}

\noindent
\textcolor{magenta}{
2) The decrease of the energy due to
hybridization with a central Cu ion with a turned spin
and surrounding oxygen ions.}
The hybridization energy matrix reads

\[
H_h = \left (
\begin{array}{ccc}
\epsilon_L+E^{(S)} & \frac{T}{\sqrt{2}}A_L \cos \alpha & 0 \\
\frac{T}{\sqrt{2}}A_L \cos \alpha & \epsilon_0 & T \sqrt{3} \\
0 & T \sqrt{3} & \epsilon_2 \\
\end{array} \nonumber
\right ) ,
\]
where diagonal elements equal to the energies of the three states
under consideration: \textcolor{blue}{ $\epsilon _L$
is the energy of the free hole wave packet in the AF-ordered
lattice with one turned spin,} \textcolor{green}{$\epsilon _0 = U \langle n
_{\downarrow} \rangle- \epsilon - \epsilon_{min} $
is the energy of the Cu-state as compared to the energy of the minimum of
the hole zone,}
\textcolor{blue}{
$$\epsilon _2 = \frac {2T^2} {U-\epsilon} - \frac 4 3 t
- \epsilon_{min} $$ is the energy of the symmetrized
oxygen states as compared to the energy
of the minimum of the hole zone}
\textcolor{red}{\cite{schneider}}\textcolor{blue}{.}
The lowest eigenvalue of this Hamiltonian
$E_h$ as well as the energy
of the free wave packet $\epsilon_L$ are
presented in Fig.2. The corresponding curves cross for
the wave packet size $L_b \approx 4a \prime$ (i.e. for the wave packet of
32 CuO$_2$ units). The crossing energy (counted from the energy
of the band minimum) is $E_b \approx 0.05 eV$ gives the energy
of the barrier.}

\noindent
The calculated value of the barrier
explains the rather long lifetime ($ \sim 1ms$)
of optically-created charge carriers.

\noindent
It also agrees with the experimental value
of the barrier for the hoping motion of the holes at
low doping \textcolor{red}{\cite{kremer}}.

%\newpage


\begin{center}
\textcolor{blue}{
\bf{Stripes}}
\end{center}

\vspace{5mm}
\noindent
Our calculations show that at finite doping the spin-polaron states
become less favorable and  at
a critical concentration $\sim 0.5$ they turn out to be metastable.
\textcolor{magenta}{
We also have found that already at small hole
concentrations phase separation of the charge carriers into domains
with a local hole concentration $c\sim 0.5 - 0.6$ takes place.
The free energy of the
domain only weakly depends on its shape.}
\textcolor{blue}{ As a result the formation of
stripe domains with local hole concentrations close to the experimentally
observed value of $c = 0.5$ can be explained.}






\vspace{10mm}
\newpage
\begin{center}
{\bf \textcolor{blue}{Conclusion.}}
\end{center}
\vspace{10mm}
\noindent
The localized and the metastable
free-hole states in CuO$_2$ are separated by a barrier of the energy
$E_b  \sim 0.05 eV$
which can reveal itself in various kinetic phenomena.
The barrier state has relatively large size ($\sim 32$ CuO$_2$
units) and it is built-up from the band states in the vicinity of the M
($\pi/2,\pi/2$) point of the Brillouin zone and from the localized Cu-state
of the turned spin.

\vspace{11mm}
\noindent
\textcolor{blue}{
The barrier of such a height allows us to explain the experimentally observed
coexistence of free and self-trapped hole states in high T$_c$
superconductors. It is also consistent with the value of
the activation barrier for the hopping motion of holes at low doping.}


\newpage
\vspace{11mm}
\noindent
At finite doping the spin-polaron states
become less favorable and  at
a critical concentration $\sim 0.5$ they turn out to be metastable.


\vspace{11mm}
\noindent
\textcolor{magenta}{At small hole
concentrations phase separation of the charge carriers into domains
with a local hole concentration $c\sim 0.5 - 0.6$ takes place.
The free energy of the
domain only weakly depends on its shape. As a result the formation of
stripe domains with local hole concentrations close to the experimentally
observed value of $c = 0.5$ can be explained.}












\newpage


\begin{thebibliography} {}

\bibitem{oles}  A.M.Ole\'{s}, and J.Zaanen, Phys. Rev.
   B {\bf 39}, 9175 (1989).

\bibitem{siebold} G.Seibold, E.Sigmund, and V.Hizhnyakov, Phys.Rev.
   B {\bf 48}, 7537 (1993).

\bibitem{schluter} M.Schluter, J.Hybertsen, and N.E.Christensen,
   Physica C {\bf 153}, 1217 (1988).

\bibitem{stechel} E.B.Stechel and D.R Jennison, Phys.Rev. B {\bf 38},
   4632 (1988).

\bibitem{schneider} V.Hizhnyakov, E.Sigmund, and Schneider, Phys.Rev.
   B {\bf 44}, 12639 (1991).

\bibitem{kremer} R.K.Kremer, A.Simon, E.Sigmund,
   and V.Hizhnyakov, Proc. of Estonian Academy of
   Sciences , Physics {\bf 44}, 274 (1995).

\end{thebibliography}}
\end{Huge}

\newpage

\begin{LARGE}

\noindent
Figure 1:
The mean hole energy $W$ in the CuO$_2$ plane
with $ N_0$ additional holes of the same spin added to the  rigid (a)
and self-consistent (b) AF lattices with $N$ elementary cells vs
the concentration of additional holes $N_0 / N$.

\vspace{1cm}

\noindent
Figure 2:
The energy $E$ of the free-hole state in the AF-ordered lattice
with one turned spin (a) and of the localized ferron state (b)
vs the size of the states $L$.

\vspace{1cm}

\noindent
Figure 3: Full ehergy of the crystall E
vs hole concentration c in the hole-rich (stripe)
region.
Initial (mean) hole concentration is 0.05.

\end{document}


\begin{thebibliography} {}
\bibitem{mott} N.F.Mott, J.Phys.:Condens.Matter {\bf 5}, 3487 (1993).

\bibitem{hizhnyakov} V.Hizhnyakov and E.Sigmund, Physica C, {\bf 156},
   655 (1988).

\bibitem{tajima} S.Tajima, H.Ishii, T.Nakahashi, T.Takagi,
   S.Ushida, and M.Seki, J.Opt.Soc.Am. B {\bf 6}, 475 (1989).

\bibitem{mihailovic1} D.Mihailovic, C.M.Forster, K.Voss and
   A.J.Heeger, Phys.Rev.B {\bf 42}, 7989 (1990).

\bibitem{sigmund} V.Hizhnyakov, E.Sigmund, and G.Zavt, Phys.Rev.
   B {\bf 44}, 12639 (1991).

\bibitem{bianconi} A.Biancconi at al., Phys.Rev.Lett.
    {\bf 76}, 3412 (1996).

\bibitem{egami} T.Egami in "Physical Properties of High
   Temperature Superconductors V" Ed.D.M.Ginsberg,
   (World Scientific, 1996).

\bibitem{pekar} M.Deigen and S.Pekar, ZETP {\bf 21}, 803 (1951) (in russian)

\bibitem{rashba} E.I.Rashba, in "Excitons", ed. E.I.Rashba and
   M.D.Sturge ch.13, (North-Holland 1982).

\bibitem{zavt} A.Shelkan and G.Zavt, Proc. of Estonian Academy of
   Sciences , Physics {\bf 39}, 358 (1990).

\bibitem{footnote} The self-trapping barrier has earlier
been estimated in
\cite{sigmund}, where it is supposed that the minimum of the hole band
is in $\Gamma$-point of the Brillouin zone. In fact, as it was already
pointed above, according to our and other recent
calculations there are 4 minima situated in the X- and
the Y-points. Besides, the energy of the wave-packet of
the free hole in \cite{sigmund} was calculated
without taking account of the correlation energy caused by the Hubbard
interaction. The account of both these factors (see the manuscript)
results in a remarkable change
($\sim 3$ times enlargement) of the value of the barrier.

\bibitem{oles}  A.M.Ole\'{s}, and J.Zaanen, Phys. Rev.
   B {\bf 39}, 9175 (1989).

\bibitem{siebold} G.Seibold, E.Sigmund, and V.Hizhnyakov, Phys.Rev.
   B {\bf 48}, 7537 (1993).

\bibitem{schluter} M.Schluter, J.Hybertsen, and N.E.Christensen,
   Physica C {\bf 153}, 1217 (1988).

\bibitem{stechel} E.B.Stechel and D.R Jennison, Phys.Rev. B {\bf 38},
   4632 (1988).

\bibitem{o-o} A.Shelkan, G.Zavt, V.Hizhnyakov, and E.Sigmund,
   Z.Phys. B {\bf 104}, 433 (1997).

\bibitem{note} We underline that there indeed exists the self-consistent
solution of the problem with one turned spin in the AF-ordered lattice
with the energy $E^{(S)} \approx 0.15 eV$ (see above);
the appearance of such solution reflects the stability
of the spin configuration with one turned spin.

\bibitem{schneider} V.Hizhnyakov, E.Sigmund, and Schneider, Phys.Rev.
   B {\bf 44}, 12639 (1991).

\bibitem{kremer} R.K.Kremer, A.Simon, E.Sigmund,
   and V.Hizhnyakov, Proc. of Estonian Academy of
   Sciences , Physics {\bf 44}, 274 (1995).

\end{thebibliography}
