\documentclass{kapproc}
\usepackage{epsfig}
%\usepackage{graphicx}
%\kluwerbib
\normallatexbib
%\bibliographystyle{apalike}
\begin{document}
\newcommand{\lesssim}{\mbox{\raisebox{-0.6ex}{$\,
    \stackrel{<}{\scriptstyle\sim}\,$}}}
\newcommand{\gtrsim}{\mbox{\raisebox{-0.6ex}{$\,
    \stackrel{>}{\scriptstyle\sim}\,$}}}

\articletitle{Anharmonic gap modes in alkali halides}

\author{D.Nevedrov}
\affil{Institute of Theoretical Physics, University of
Tartu, T\"{a}he 4, 51010 Tartu, Estonia}

\author{V.Hizhnyakov}
\affil{Institute of Theoretical Physics, University of
Tartu, T\"{a}he 4, 51010 Tartu, Estonia}
\email{hizh@eeter.fi.tartu.ee}

\author{A.J.Sievers}
\affil{Laboratory of Atomic and Solid State Physics and Materials Science
Center, Cornell University, Ithaca, New York 14853-2501, USA }

\begin{keywords}
lattice dynamics, localized states, relaxation
\end{keywords}



%\maketitle


\begin{abstract}


%\noindent
We calculate the frequencies of the anharmonic gap modes in alkali
halide crystals, in dependence of their
%The calculations illustrate the frequency of these modes as a function of
amplitude. The lifetime of the modes, caused by
emission of two acoustic phonons, is also calculated basing on new
nonperturbative theory of multiphonon transitions.

\end{abstract}


%\pacs{PACS numbers: }


%\section{Introduction}
It is a well known fact that stable localized vibrations exist in classical
anharmonic lattices, \cite{dolgov,sivtak,flach}; the
phenomena has been confirmed by numerous numerical studies based on the
integration of classical equations of motion. The observation of such modes
in three dimensional lattices brings certain computational difficulties.
Therefore anharmonic local modes (ALM) were studied only in few real 3D
crystals (see, e.g. \cite{kisgap}, where existence of anharmonic gap
modes  in ${\rm NaI}$ have been demonstrated).
Here we apply a new method (see\cite{electronic}), which
allows one to reduce the nonlinear problem of ALMs to
a linear inverse problem of phonons scattering on a
local potential. By this method
we calculate anharmonic gap modes in alkali halide
crystals. Phonon spectra in these crystals are
calculated in the shell model \cite{bilz};
anharmonic constants are determined from Born-Coulomb-Mayer potentials.


The ALM are stable only in the classical limit; account
of zero-poit fluctuations of the lattice leads to the multiphonon decay
of these modes. The peculiarity  of the problem is that, due to large
amplitude of the mode,
standard perturbation theory is not applicable for calculation
of the decay rate. To solve the problem, we
apply the nonperturbative theory of two-phonon decay,
developed in  \cite{hizhrev,hizhnev}.

%\section{Anharmonic gap modes in alkali halides}

We consider an ALM, localized on a light ion
in a alkali halide crystal with the gap in phonon spectrum.
In the central forces approximation the effect of the ALM  on
phonons is described by the potential operator
\begin{equation}
V  =  \sum_n \bigg[
\frac{1}{2} V_2 \hat{U}^2_n +
\frac{1}{3!} V_3  \hat{U}^3_n  +
\frac{1}{4!} V_4  \hat{U}^4_n + \ldots \bigg],
\nonumber \\
\end{equation}
where  $V_m$ is the $m$-th derivative of the
pair potential between nearest ions,
$\hat{U}_n$ is  the displacement
operator of the  nearest  neighbour (nn) $n$ in direction of the central ion
and with respect to it.
We take  $\hat{U}_n=Q_n (t) + \hat{q}_n -
\hat{q}_{0n}$, where $Q_n (t)\simeq A_n \cos{\omega_l t} + \xi_n$
stands for the displacement of the
classical ALM of frequency $\omega_l$,
$\hat{q}_{0n}$ and $\hat{q}_n$  are the  displacement
operators of
the central ion and ion $n$ along the $0-n$ direction;
the constant $\xi_n$ accounts for the DC component of the ALM.
The equation for $\hat{q}_{n}$ reads:
\begin{eqnarray}
M_n \ddot{\hat{q}}_{n}= -
\sum_{n'} ( V_{2 n n'}+W_{2 n n'} (t)) \hat{q}_{n'},
\label{eq:b}
\end{eqnarray}
where $V_{2 n n'} = V_{0nn'} + \tilde{V}_{nn'}$,
$\tilde{V}_{ n n'}$ and
$W_{n n'}(t) =  2 v_{n n'}\cos{\omega_{l} t} + \ldots$
are the independent and the dependent on time contributions of the ALM to
the springs matrix, $M_n$ is the mass of the ion $n$.
According to \cite{electronic}
%, a ALM is stable with respect to an infinitesimal change of its phase,
%only if
the frequency  $\omega_l$
is present in the disturbed
phonon spectrum. This allows one to calculate the ALM by a self-consistent
consideration of the local dynamics, described by the
Eq.(\ref{eq:b}) in the rotating wave approximation $W(t)=0$.
The nonzeroth elements of the matrix $\tilde{V}_{nn'}$ equal
\begin{equation}
\gamma_{\alpha} = \frac{1}{4}  A^2_{\alpha} {\Big \{ } V_4 +
\frac{V_3^2}{M_1} (G_{11}^{(0)}(0)-G_{11'}^{(0)}(0)) {\Big \} },
\label{eq:gamma}
\end{equation}
$A_{\alpha}$ is the $\alpha$'s component of the amplitude of the
ALM,
$G^{(0)}(\omega)$  are Greens functions the nn ions
in perfect lattice
(higher order terms with respect to $A^2$ are neglected).
The second, negative term in the wavy brackets
in (\ref{eq:gamma}) in its absolute value  $\sim 5$
times exceeds the first term $V_4$. This means that the ALM causes
softening of local springs. It is why
ALMs appear only in the gap of the phonon spectrum \cite{kisgap}.
We calculate them by applying Lifshitz method (see Figs. 1,2,3).
One sees that  the minimal amplitude of the gap ALM is $0.2 - 0.3 \AA$
for vibrations in (100)
direction,  $0.3 - 0.45 \AA$ in (110) direction and $0.4 - 0.6 \AA$
in (111) direction (in {\rm NaI}  minimal $A$ in (111) direction
$ \approx 0.55 \AA$, in agreement with \cite{kisgap}).

%\section{Quantum decay of ALM}
The time-dependent terms in (\ref{eq:b})
are responsible for the two-phonon  relaxation \cite{hizhrev}.
The rate of the energy loss by the ALM
equals
\begin{equation}
\frac{dE_l(t)}{dt} \simeq -\frac{\hbar \omega_l}{4 \pi}
 \int_0^{\infty} \!\!\!\! d \omega \mbox{Sp}
(P(\omega_l -\omega )vP(\omega)v)(1+2n(\omega ))
\end{equation}
where $n(\omega)=1/(e^{\hbar\omega/kT}-1)$, $v_{\alpha}
= V_3 A_{\alpha}/2 M_1$,
 \begin{equation}
P(\omega ) =  \lbrace I-G(\omega)vG^*(\omega_l-\omega )v\rbrace ^{-1}
ImG(\omega).
\end{equation}
The results of calculations of the gap ALM in {\rm NaI} are presented
in Fig. 4 and 5. One sees that
most stable are gap ALM in (111) direction.

The research was supported by ESF Grant No 3864 and by NRC Grant of
the American Academy of Sciences.

\begin{chapthebibliography}{3}

\bibitem{dolgov}
A.S.Dolgov, Fiz. Tverd. Tela (Leningrad) {\bf 28}, 1641 (1986)
[Sov. Phys. Solid tate {\bf 28}, 907 (1986)].
\bibitem{sivtak}
A.J.Sievers, S.Takeno, Phys. Rev. Lett. {\bf 61}, 970 (1988).
\bibitem{flach}
S.Flach, C.R.Willis, Physics Reports {\bf 295},181 (1998).
\bibitem{kisgap}
S.A.Kiselev, and A.J.Sievers, Phys. Rev. B {\bf 55}, 5755 (1997).
\bibitem{electronic}
D.Nevedrov, V.Hizhnyakov, E.Sigmund,
xxx.lanl.gov, cond-mat/9908268.
\bibitem{bilz}
H.Bilz, W.Kress, {\it Phonon Dispersion Relations in
Insulators} (Springer, Berlin, 1979).
\bibitem{hizhrev}
V.Hizhnyakov, Phys.Rev. B {\bf 53}, 13981 (1996).
\bibitem{hizhnev}
%V.Hizhnyakov, D.Nevedrov, Proc. Estonian Ucad. Sci. Phys. Math. {\bf 44},
%376 (1995);
V.Hizhnyakov and D.Nevedrov, Phys. Rev.      1997;
Z. Phys. Chem. {\bf 201}, 301 (1997);
Pure and Appl. Chem. {\bf 69}, 1195 (1997).

\end{chapthebibliography}

\bigskip

\noindent
{\bf Figure 1.}  Frequency of the anharmonic mode ($\omega_l/\omega_m$)
versus amplitude (in $\AA$) in {\rm KI}, $\omega_m$ here is the lowest
frequency of the optic band.

\noindent
{\bf Figure 2.} The same as in Fig1, but for {\rm NaI}.

\noindent
{\bf Figure 3.}  The same as in Fig1, but for {\rm RbF}.

\noindent
{\bf Figure 4.}  Relaxation rate (in $10^{12} sec^{-1}$) versus gap mode
amplitude (in $\AA$) in $KI$.

\noindent
{\bf Figure 5.}  Gap mode
amplitude (in $\AA$) versus time (in periods $2\pi /\omega_m$) in $NaI$.


\end{document}
