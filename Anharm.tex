\documentstyle[aps,prb]{revtex} 
\title{Anharmonic local modes in perfect and imperfect crystals} 
\author{ D.Nevedrov$^{1}$,V.Hizhnyakov$^{2,3}$, A.J.Sievers$^3$} 
\address{$^1$$NORDITA, Blegdamsvej 17, Copenhagen DK-2100, Denmark.\\
$^2$Institute of Theoretical Physics, 
University of Tartu, T\"{a}he 4, Tartu EE2400, Estonia;
Institute of Physics, University of Tartu, Riia 142, Tartu EE2400, Estonia.\\ 
$^3$Laboratory of Atomic and Solid State Physics and Materials Science 
Center, Cornell University, Ithaca, New York 14853-2501, USA } 

\begin{document} 
\begin{abstract} 
Local modes  of large amplitudes in nonlinear lattices are considered. 
The applied method allows the reduction of nonlinear problem to a linear 
inverse problem of phonons scattering on a local potential. 
As an example, the method is applied for analytical description of ALMs
in monatomic chain.
Simple analytical expression for the softening of the
local springs by an odd anharmonic mode, which accounts for 
the long-range lattice relaxation, is obtain  and applied for  
calculation of anharmonic gap modes in pure and 
impure alkali halide crystals. It is found that at large amplitudes 
the gap modes became unstable.
\end{abstract} 

                                                                        
\section{Introduction} 
Recent theoretical studies of nonlinear vibrational dynamics lead to 
prediction of  long-living localized anharmonic modes (ALMs)
in perfect anharmonic lattices
\cite{dolgov,sivtak,page,kiselev,zavt} (see also review 
\cite{pagesievers,flach} 
and references therein). The main approach, which have been used  for
investigation of these novel excitations, is based on direct numerical 
integration of  classical equations of motions of finite lattices (usually 
with the periodical boundary conditions).
The approach is rather efficient in the case of simple 1D  lattices, 
but requires rather expensive computations in the case of
2D  and 3D lattices due to the
fast growth of the number of numerical operations and computational
time with increase of the number degrees of freedom. Therefore ALMs
only in few real 3D crystals were studied so far
(see, e.g. \cite{kisgap}, where existence of anharmonic gap
modes  in ${\rm NaI}$ have been demonstrated). On the contrary, 
in harmonic approximation calculations can be performed also 
for macroscopically large pure and locally distorted crystals of any
dimension.  

Here we apply a method which makes use of harmonic approximation results
for calculations of the anharmonic local modes (ALM). The method 
allows one to reduce the nonlinear problem of ALMs to a properly 
formulated problem of linear local dynamics. It is applicable for lattices of 
arbitrary dimension. The method is efficient in the case of strongly 
localized modes, when it allows one to obtain analytical solutions. Our 
consideration is based on classic mechanics; account of quantum effects,
causing multiphonon decay of the ALMs, see in \cite{hizhrev,hizhnev}. 

The starting points for the method are the stability conditions of an
ALM with respect to small fluctuations:
\begin{itemize}
\item
All frequencies of the spectrum of these fluctuations should be real and  
positive,  
\item
Spectrum of fluctuations should contain also the basic frequency $\omega_l$ of
the ALM, i.e. there should be harmonic local mode (HLM) with the same 
frequency $\omega_l$, 
\item
The ratios of the amplitudes of atoms of the HLM should coincide with that of 
corresponding ALM.  
\end{itemize}

The physical meaning of the first condition  is obvious: the configuration 
of the crystal lattice with the ALM must be stable (not meta-stable). 
The last two conditions ensure that a solution, which describes  
an ALM are stable  with respect to infinitesimal changes of the ALM phase. 
They allow one to perform self-consistent calculations
of the ALMs. This can be done as follows:
choosing a trial shape of the ALM, we can determine the changes
of the elastic springs, caused by the ALM \cite{kisgap}. 
Then, by applying  the Lifshitz formula
of local dynamics in harmonic approximation, the expressions for the
HLM (and ALM) may be obtained, both for its frequency and amplitudes of 
contributing atoms. These expressions allow the obtaining
of the self-consistency conditions. The method  reduces the nonlinear 
problem of 
calculation of an ALM to a kind of a linear inverse problem of phonons 
scattering on a local potential. If the ALM is strongly localized, then 
the procedure is simple while one needs to calculate only few amplitudes. 
This can be done analytically. 


To check the method, we apply it first to the ALMs in monatomic chain 
with hard quartic anharmonicity. We will show that one gets
already known  results \cite{sivtak,page}.
Then we will apply it to odd ALMs in pure and impure  
alkali halide crystals with light and heavy ions and with account of both,  
cubic and quartic anharmonicity (the ALMs under consideration are almost fully
localized on the light atoms). Phonon spectra in these crystals are 
calculated in the shell model \cite{shell,bilz,kristofel}.
Anharmonic constants are determined from the Born-Coulomb-Mayer-Van der
Waals potentials \cite{potential}. 

\section{General} 
Let us consider a vibrational system with the potential energy  
\begin{equation} 
{V} = \frac{1}{2} \sum_{n_{1} n_{2}}\!{V^{(2)}_{n_{1} n_{2}} 
{U}_{n_{1}} {U} _{n_{2}}} 
+ \frac{1}{3}\! \sum_{n_{1} n_{2} n_{3}}\!\!{V^{(3)}_{n_{1} n_{2} n_{3}} 
{U}_{n_{1}} {U}_{n_{2}} {U}_{n_{3}}} + \ldots\,, 
\label{eq:v} 
\end{equation} 
where ${U}_{n}$ are the Cartesian displacements of atoms situated at the site 
$n$, $V^{(2)},\, V^{(3)},\ldots$ are harmonic and anharmonic springs. The 
displacements $U_n$ satisfy the following equations of motion 
$ - M_n \ddot{U}_n = dV/dU_n$.
where $M_{n}$ are the masses of atoms. We suppose that a ALM is excited
at the site $n=0$ and its nearest neighbors. This excitation is the solution 
of the equations of motion with exponentially decreasing  $|U_n|$ 
with $|n| \rightarrow \infty$. As it is known \cite{sivtak,page} a
solution, which corresponds to the ALM with the frequency  $\omega_l$ has the 
form $U_n(t)= A_n \cos(\omega_l t) + \xi_n + O(\omega_l)$, where $O(\omega_l)$ 
is the sum of small terms of the frequencies $3\omega_l$,  
$5\omega_l, \ldots$ (i.e. the higher order harmonics; these harmonics are 
neglected below).  The amplitudes $A_{n}$ and the shifts $\xi_{n}$ 
(s.c. dc-components of the ALM) can be 
found by equating the coefficients before the terms with the same time 
dependence \cite{sivtak,page,flach}. 

To take into account small fluctuations we present the displacements in the  
form $\bar{U}_n = U_n + q_n/ \sqrt{M_n}$, where $q_n $ stands for the reduced  
small displacement. The equations for ${q}_{n}$ read: 
\begin{eqnarray}
\ddot{q}_{n}=  
\sum_{n'}{{\big (}V_{2 n n'}+W_{2 n n'}{\big )}{q}_{n'}},
\label{eq:b}
\end{eqnarray}
where $V_{2 n n'}$ is the dynamical matrix in harmonic approximation, 
\begin{equation}
W_{2 n n'} = \frac{2}{\sqrt{M_{n} M_{n'}}} {\Big (} 
\sum_{n_{1}} V^{(3)}_{n n' n_{1}}\xi_{n_{1}} + 
\frac{3}{4} \sum_{n_{1} n_{2}} V^{(4)}_{n n'n_{1} n_{2}} 
A_{n_{1}} A_{n_{2}} + \ldots {\Big )} \label{eq:c}
\end{equation}
(here the averaging of $\sim \cos{\!^2\!(\omega_l t)}$
over vibrational period is made; the
time-dependent terms $\sim \cos{k \omega_{l} t}$, $k=0,1,2,\ldots$ are 
neglected; these terms are important in quantum mechanical description
of fluctuations being responsible for the multiphonon decay of the ALM
\cite{hizhrev,hizhnev}).  

 
The shifts $\xi_n$ result from the lattice relaxation, caused by the
forces, being  associated
with the ALM (these forces differ from zero in the lattices 
with non-zero odd anharmonicities). The relaxation of ions position 
in the lattice extends far away, even if 
the forces appear locally (in 3d lattices the shifts decrease as the 
distance in the power $-3$; in 1d lattices they do not decrease 
with the distance at all). This effect is essential to 
account also in the case, when  contribution of few local 
shifts to the springs $W_{2nn'}$ is taken into consideration, while 
result for these shifts essentially depends on, whether the surrounding
ions are allowed to shift, or not. 
To account this effect, we expand the shifts over the phonon displacements 
$x_{io}$: 
$\xi_n= \sum_i e_{ni} x_{oi}$
(in a perfect lattice $e_{in} \sim e^{(i\vec{k} \vec{n})}$, where $\vec{k}$
is the wave vector).  The displacements $x_{0i}$  can be found
if one adds the linear term   $\sim [\sum_{n n' n''} M_{n''}^{-1/2}\sum_i
 V^{(3)}_{n n' n''} A_n A_{n'} e_{ni} + \ldots] x_i$ with respect to 
phonon coordinates $x_i$, to phonon Hamiltonian
$H_{ph} = H_{0,ph} + \tilde{V}_2,$, where $H_{0,ph}$
is the phonon Hamiltonian of the harmonic lattice, 
$\tilde{V}_2  = (1/2) \sum_{nn'}W_{2 n n'}q_n q_{n'} $
describes the effect of the ALM to the phonons.  The shifts are equal
\begin{equation}
\xi_{n_{1}}\approx  \sum_{n'_{1} n_{2} n_{3}}
 (M_{n_{1}} M_{n'_{1}})^{-1/2} V^{(3)}_{n'_{1} n_{2} n_{3}} 
A_{n_{2}} A_{n_{3}}
G_{n_{1} n'_{1}}(0) + \ldots. \label{eq:xi}
\end{equation}
Here and below 
$$
G_{n n'} (\omega) = \sum_j \frac{e_{n_{j}} e_{n'_{j}}}{(\omega -
i \varepsilon)^2 - \omega_{j}^2}
$$
is the spectral Green's function of phonons
\cite{economou,maradudin} (if needed, $G_{n_{1} n'_{1}}(0)$ 
 should be determined self-consistently).
Inserting Eq. (\ref{eq:xi}) into Eq. (\ref{eq:c}) one gets the expression 
for the  renormalization of the dynamic matrix due to the ALM. The term
$\sim G_{n_1 n'_1}(0)$, then describes the softening of the local springs,
caused by cubic (and higher order odd) anharmonicity(is).  

The HLMs, induced by the ALM, are determined by the poles of the spectral 
Green's function. 
To find this  matrix-function, we apply Lifshitz formula 
\begin{equation}
G_{nn'}(\omega) = G_{nn'}^{(0)}(\omega) + \sum_{n_1 n_2} 
G^{(0)}_{nn_1}(\omega) W_{2n_1 n_2} 
G_{n_{2} n' }(\omega),
\end{equation}
where $G_{nn'}^{(0)}(\omega)$ are the Green's functions of the perfect 
lattice. These functions can be calculated by standard methods of lattice 
dynamics \cite{maradudin}. 
The ALM is stable if the Green's function does not have poles 
on the imaginary axis of the complex $\omega$-plain.


To find the frequency and shape of the HLM (and ALM) let us 
introduce the 
configurational coordinates $Q_{\nu} = \sum_{n} S_{\nu n} q_n,$ and  
$\tilde{Q}_{\mu} = \sum_{\nu} s_{\mu \nu} (\omega) y_{\nu}$. The first 
transformation is chosen to diagonalize the perturbational quadratic form 
$\tilde{V}_2$: $\tilde{V}_2 = \sum_{\nu} \eta_{\nu} Q_{\nu}^2/2.$
The second transformation diagonalizes the dimensionless 
Green's function matrix $\check{G}_{\nu \nu'}=
\sqrt{\eta_{\nu} \eta_{\nu'}}\bar{G}^{(0)}_{\nu\nu'}(\omega)$, 
($\bar{G}^{(0)}_{\nu\nu'}(\omega)$ are the lattice Greens functions  
in $Q_{\nu}$-space): 
$\tilde{G}^{(0)}_{\mu \mu'}(\omega) = \delta _{\mu \mu'} 
\sum_{nn'} R_{\mu n}R_{\mu n'} G^{(0)}_{n n'} (\omega)$; here
$R_{\mu n} (\omega) =  
\sum_{\nu} s_{\mu \nu}(\omega) \sqrt{\eta_{\nu}} S_{\nu n}$. 
In the $\tilde{Q}_{\mu}$-representation the Lifshitz formula gives 
$\tilde{G}_{\mu \mu}(\omega) =  
{\tilde{G}^{(0)}_{\mu \mu}(\omega)}{(1 -\tilde{G}^{(0)}_{\mu \mu}(\omega))}.$

From symmetry considerations one can chose the configurational 
coordinate $\tilde{Q}_{\mu}$ which corresponds to the ALM under investigation. 
Frequency of the ALM (and of the HLM) satisfies the relation
\begin{equation} 
\tilde{G}^{(0)}_{\mu \mu}(\omega_l) = 1, \label{eq:G} 
\end{equation} 
while the amplitude (not normalized) of the $n$' atom in the HLM (and in the 
ALM) is given by the formula: 
\begin{equation}
a_{\mu n} = \frac{R_{\mu n}(\omega_l)}{\sqrt{|G^{(0)'}_{\mu \mu} (\omega_l)|}},
\label{eq:d}
\end{equation} 
where  
$G^{(0)'}_{\mu \mu}(\omega) = d G^{(0)}_{\mu \mu}(\omega)/ d \omega$. 
Using normalization $\sum_n a_{\mu n}^2 = 1$ we obtain: 
\begin{equation} 
a_{\mu n} = 
\frac{R_{\mu n}(\omega_l)}{\sqrt{|\sum_{\nu} s_{\mu \nu}(\omega_l)^2 
\eta_{\nu}|}}. 
\label{eq:e}
\end{equation} 
The relations (\ref{eq:e})  is the self-consistency condition, 
which gives the set of equations for $a_{\mu n}$. 
In combination 
(\ref{eq:G}) they allow one to find the amplitudes $A_{n}$ and shifts
$\xi_n$ of the ALM of 
given frequency $\omega_l$. The method works better for strongly localized 
ALMs with high frequency, when the number of contributing amplitudes $A_n$
(and equations to be solved) is small. 

In the case of high frequency of the ALM  ($\omega_l \gg \omega_M$, $\omega_M$ 
is maximal phonon frequency) the nondiagonal elements of the  
$G^{(0)}_{nn'}(\omega_l)$ are much smaller than the diagonal ones. Then 
$\check{G}^{(0)}_{\nu \nu'}(\omega_l) \approx \delta_{\nu \nu'}  
G^{(0)}_{00}(\omega_l)$, i.e. the Greens-function matrix 
is diagonalized simultaneously with $\tilde{V}_2$. In this case the normalized
amplitudes of the ALM can be determined by simple self-consistency 
condition $a_n =  S_{\nu n}$; absolute amplitudes one can  find from the
relation $G_{\nu \nu}(\omega_l) = \eta_{\nu}^{-1}$. 

If a immobile ALM is considered and  the limited number of atomic
displacements is taken into consideration
then the quadratic form
$\tilde{V}_2$ does not  depend on the totally-symmetric coordinate. 
This means that all displacements, which contribute to the diagonalized 
quadratic form
$\tilde{V}_2$, are orthogonal to this coordinate. This means that
the sum of displacements of all atoms, which contribute to the ALM,
should be equal zero: $\sum_n A_n = 0$. The same holds for the sum of 
moment of atoms at any time-moment. This property of the ALMs stands for the
immobility condition.  

\subsection{Even and odd ALMs in monatomic chain} 
First, in order to check the method, we apply it to description of
ALMs in monatomic chain. The potential 
energy of the chain with account of the nearest-neighbor interactions has the 
form 
$$ 
V=\sum_n \sum_r (V^{(r)}/r) (U_{n+1} -U_n)^r. 
$$ 

In harmonic approximation ($V^{(2)}>0,\,\, V^{(r)}=0,\,\, r\geq 3$) for 
$\omega/\omega_M > 1$ the Green's functions of the chain equal \cite{economou}
$$
G^{(0)}_{nn'}(\omega) = \frac{\omega_M (-\rho)^{|n-n'|}}
{16 \omega V^{(2)} \sqrt{\omega^2 /\omega_M^2 -1}},
$$ 
$\omega_M = 2\sqrt{V^{(2)}}$ is the top phonon frequency,
$\rho = (\omega/\omega_M - \sqrt{\omega^2/\omega_M^2-1})^2 < 1$.
In the limit of strong amplitudes and high frequency one gets two strongly 
localized  ALMs \cite{sivtak,page}: 
\begin{enumerate}
\item the odd mode with $A_n= A_{-n}$,  
$A_1 \simeq -A_0/2$, $|A_n|\ll |A_1|$, $n \ge 1$ and 
\item the even mode with 
$A_{n+1} = -A_{-n}$, $|A_{n+1}| \ll |A_0|$, $n \ge 1$. 
\end{enumerate}
However only the even mode is stable with respect to small
fluctuations \cite{kissievers}.
Below we give analytical description of these modes
in the chain with quartic($r=4$) anharmonicity.
In this case the poles of the Green's function can shift only to higher
$\omega$ values. Therefore $Im G(i |\omega|) = 0$ and the ALMs are stable.

\subsubsection{Even mode}
Let us consider first the even mode. With account of four central atoms 
displacements the contribution of the ALM to the potential energy of phonons 
has the form 
$$
\tilde{V}_2 = K_4 A_0^2 [b((q_2-q_1)^2+(q_0 -q_{-1})^2)+(q_1 - q_0)^2], 
$$
where $K_4= 6V^{(4)}$, $b=(1+\beta)^2/4$, $\beta = A_2/A_0$.  
Two following even modes contribute to the $\tilde{V}_2 $ and ALM: 
$$
y_1 = {q_0-q_1}{\sqrt{2}},\quad y_2 = {q_2-q_{-1}}{\sqrt{2}}.
$$
In the strong localization limit the main contribution is given by $y_1$. 
Contribution of $y_2$ depends on the $\omega_l$ being larger for smaller 
$\omega_l$. Our task is to describe the last dependence. In $y_1,\,y_2-$ 
subspace  $\tilde{V}_2 = K_4((2+b)y_1^2 - 2b y_1y_2 + b y_2^2)$. 
This quadratic form is diagonalized in the rotated basis on the angle
$\phi =\arctan{(b)}/2$.
In the limit of high frequency $\omega_l$ the self-consistency condition  
reads 
$\sin{\phi}=\beta \approx b/2 = (1+\beta)^2/8.$
One gets 
$\beta \approx 1/6$ in agreement with the corresponding calculation of $\beta$
on the basis of the nonlinear equations of motion \cite{page}. 

To describe $\omega_l-$dependence of $\beta$ we should find coordinates 
$\tilde{Q}_{1,2}$ which diagonalize the Green's function matrix 
$\check{G}^{(0)}_{\nu \nu'}= 
G^{(0)}_{00}(\sqrt{\eta_{\nu}\eta_{\nu'}}g_{\nu\nu'})$, $\nu, \nu' =1,2$,
$\eta_{1,2} =K_4 A_0^2 (1+b \pm \sqrt{1-b^2})$,
\begin{eqnarray}
g_{11} &=&\rho_{11}\cos^2{\phi} +
\rho_{22}\sin^2{\phi} + \rho_{12} \sin{(2\phi)}, \nonumber \\
g_{22} &=&\rho_{11}\sin^2{\phi} +
\rho_{22}\cos^2{\phi} - \rho_{12} \sin{(2\phi)}, \nonumber \\
g_{12} &=&-\rho_{12} \cos{(2\phi)} + (\rho_{11}- \rho_{22})
\sin{(2\phi)}/2,  \nonumber
\end{eqnarray}
where
$\rho_{11} = 1+\rho$, $\rho_{22} = 1+\rho^3$, $\rho_{12}=\rho(1+\rho)$.
The diagonalization is achieved by additional rotation on angle
\begin{equation}
\alpha=\arctan{( 2\sqrt{\eta_1 \eta_2} g_{12}
/(\eta_1 g_{11}-\eta_2 g_{22}))}/2.
\label{eq:f}
\end{equation}
The self-consistency condition now is 
$\beta = \sin{(\alpha +\phi)}$.
In the strong localization limit it gives     
$\beta \approx (1+5\rho/2\sqrt{2}(+\rho))/6$.
The dependence of the dimensionless frequency of the mode 
$\Omega_l = \omega_l/\omega_M$ on the amplitude $A_l$ is given by the equation
(\ref{eq:d}) with $\nu =1$, where
\begin{equation} 
\tilde{G}^{(0)}_{11}(\omega_l)= K_4 A_0^2G^{(0)}_{00}(\omega_l) 
{\Big (}1+b + \sqrt{1-b^2}{\Big )}
{\Big [}g_{11}\cos^2{\alpha} + g_{22} \frac{\eta_2}{\eta_1}\sin^2{\alpha} + 
g_{12}\sqrt{\frac{\eta_2}{\eta_1}}\sin{(2\alpha)}{\Big ]}. 
\end{equation}
For $\omega_l > 1.15 \omega_M$ given formulas describe the ALM
very well: the contributing amplitudes of the next nearest atoms is less 
than $10^{-2}$, there-at their contribution to the energy of the ALM is less 
than $10^{-4}$. 

\subsubsection{Odd mode}
Let us consider now the odd ALM with account of displacements of five atoms. 
The mode under consideration satisfies the parity and immobility conditions 
$A_n=A_{-n}$, $A_2+A_1+A_0+A_{-1}+A_{-2}=0$, which give 
$A_1=-A_0 (1/2 + \beta)$, $\beta = A_2/A_0$. The contribution of the ALM to 
the potential energy of phonons equals 
$$ 
\tilde{V}_2 = \bar{K}_4 A_0^2 {\Big [}
\bar{b}((q_2-q_1)^2 + (q_{-2} - q_{-1})^2) +  
(q_1-q_0)^2 + (q_{-1} -q_0)^2{\Big ]}, 
$$ 
where $\bar{K}_4=9V^{(4)}(1+2\beta/3)^2/8$, 
$\bar{b}=(1+4\beta)^2/(3+2\beta)^2$. Two odd modes, which give contributions 
to the ALM read: 
$$ 
y_1=(2q_0 -q_1-q_{-1})/\sqrt{6},\,\,\, 
y_2=(3q_2 + 3q_{-2} - 2q_0 - 2q_1- 2q_{-1})/\sqrt{30}. 
$$ 
In the $y_{1,2}$-space  $\tilde{V}_2 = \bar{K}_4 [(3+\bar{b}/3)y_1^2 -  
(2\sqrt{5}/3)\bar{b} y_1 y_2 + (5/3) \bar{b} y_2^2]$. This quadratic form is 
diagonalized in the rotated basis on the angle  $\phi \approx
\arcsin{(b/2)}$,
where $b = 2\sqrt{5}\bar{b}/(9-4\bar{b})$. In the limit
of large $\omega_l/\omega_M$ the main contribution to the ALM is given by
the $y_1$-mode; contribution of the $y_2$-mode is given by the self-consistency
condition 
$\beta \approx 3\sin{(\phi)}/ 2\sqrt{5} \approx \bar{b}/6$. Taking 
$\bar{b} \approx 1/9$ one gets the value $\beta = 1/54$ in agreement 
with \cite{page}. A more 
accurate approximation is $\beta \approx 3/131$. 

The $\omega_l-$dependence of 
$\beta$ and $A_0$- dependence of $\omega_l$ can be found in the same way 
as for the even mode. In this case 
$$
\eta_{1,2} = \frac{K_4A_0^2}{2\sqrt{5} + 4b} (3\sqrt{5} +15 b + 
(3\sqrt{5} + 9 b))(1 \pm \sqrt{1-b^2}).
$$
For small $\beta$ one gets
$
\eta_1 \approx K_4A_0^2 (48 - 40\beta)/18,
\quad \eta_2 \approx \eta_1 (4+ 27 \beta)/64.
$
The factors $g_{\nu \nu'}$ are also determined by the same expressions 
as in the even ALM case but
with the following $\rho_{\nu\nu'}$:
\begin{eqnarray}
\rho_{11} &=& 1+ \rho (4 +\rho)/3,\nonumber \\
\rho_{22} &=& 1+ \rho (4 -8 \rho + 
12 \rho^2 + 9 \rho^3)/ 15,
\nonumber \\
\rho_{12} &=& \rho (5 + 8\rho + 3\rho^2)/15. \nonumber
\end{eqnarray}
The self-consistency condition in this case is:
\begin{equation}
\beta =\frac{3 \sin{(\phi+\alpha)}}{2\sqrt{5} \cos{(\phi+\alpha)} 
-2\sin{(\phi+\alpha)}},
\end{equation}
where $\alpha$ is determined by (\ref{eq:f}); there-at  expressions for
$\eta_{1,2}$  and for $\rho_{\nu\nu'}$ are given above,
$g_{\nu\nu'}$ are also determined by the same formulae as in the even ALM case.

If $\omega_l > 1.15 \omega_M$ then the odd ALM is well localized 
($\beta < 0.065$),  more than 99.98 percent of energy of the mode 
come from 4 central atoms. 

The calculated dependences of $\Omega_l$ and $\beta$ on amplitude
for even and odd modes are plotted on Fig.1.



%%%%%%%%%%%%%%%%%%%%%%%%%%
%%%%%%%%%%%%%%%%%%%%%%%%%%%%%
%%%%%%%%%%%%%%%%%%%%%%%%%%


\section{ALMs in alkali halides} 
We present here calculations of odd anharmonic local modes in a simple
cubic lattice. We consider the case, when the ALM is strongly localized 
on the central ion.  To describe  the effect of the ALM  on the lattice 
vibrations, we take into account the pair interactions of the
central ion with all other ions, described by the potential operator
\begin{equation} 
{V} = \sum_{\alpha}\sum_{\vec{n}}\sum_{m=1}^{\infty} \frac{1}{m} V^{(m)}_{ 
\vec{n}_{\alpha}} {\big (} {R}_{\vec{n}_{\alpha}} {\big )}^{m}\,. 
\end{equation} 
Here $\alpha = x,\:y,\:z$ are the directions of crystal axes,  
$\vec{n} = (n_{x},n_{y},n_{z})$ is the vector of the lattice sites, 
$\vec{n}_{\alpha}$ is the vector of the site nearest to $\vec{n}$ in $\alpha$ 
direction, $V^{(m)}_{\vec{n}\alpha} = V_m$ is the $m$-th derivative of the 
pair potential between atoms (ions) $\vec{n}$ and $\vec{n}_{\alpha}$ at their 
distance $R_{0\vec{n}_{\alpha}}$ 
\begin{equation} 
{R}_{\vec{n}_{\alpha}} = \sqrt{(R_{0 n_{\alpha}} + {r}_{\alpha   
\vec{n}_{\alpha}})^{2} 
+{\rho}_{\alpha \vec{n}_{\alpha}}^{2}} - R_{0\vec{n}\alpha} 
\end{equation} 
is the distance between the nearest neighbors in the $\alpha$- 
direction, ${r}_{\beta\vec{n}_\alpha}=q_{\beta\vec{n}}-q_{\beta  
\vec{n}_{\alpha}}$, $q_{\beta}$ is the $\beta$-component of the displacement  
vector $\vec{q}_{\vec{n}}$ of the atom $\vec{n}$,  
$\rho^{2}_{\alpha \vec{n}_{\alpha}}    
= r^{2}_{\vec{n}_{\alpha}} -
{r}_{\alpha \vec{n}_{\alpha}}^{2}$

${r}_{\vec{n}_{\alpha}}^{2} = {r}^{2}_{x\vec{n}_{\alpha}} + 
{r}^{2}_{y\vec{n}_{\alpha}} + {r}^{2}_{z \vec{n}_{\alpha}}$. 
By expanding ${V}$ in the power series of displacement operators  
${r}_{\alpha\vec{n}_{\alpha}}$one gets 
\begin{eqnarray}  
V & = & \sum_{\alpha ,\vec{n}_{\alpha}}
\frac{1}{2}  
[V_2 r^2_{\alpha \vec{n}_{\alpha}} +  
V'_2 \rho^2_{\alpha \vec{n}_{\alpha}}  + 
\frac{1}{3} V_3{r}^3_{\alpha n_{\alpha}} + \nonumber\\ 
 & & 
V'_3 r_{\alpha n_{\alpha}} 
\rho^2_{\alpha\vec{n}_{\alpha}}+ 
\frac{1}{12} V_4 r^4_{\alpha n_{\alpha}}  
\frac{1}{2} V'_4 r^2_{\alpha n_{\alpha}}{\rho}^2_{\alpha n_{\alpha}} +
\frac{1}{4} V'' \rho^2_{{\alpha n_{\alpha}}} + \ldots],
\nonumber \\ 
\end{eqnarray} 
where 
\begin{eqnarray} 
V'_2 &=& V_1 R_0^{-1},\quad V'_3 = (V_2 - V'_2)R_0^{-1}, \nonumber \\ 
V'_4 &=& R_0^{-1}V_3 - 2R_0^{-2}(V_2 - V'_2),\quad 
V''_4 = R_0^{-2} (V_2 - V'_2), 
\end{eqnarray} 
$V_{2}$, $V_{3}$ and $V_{4}$ make account of the central, while $V'_{2}$, 
$V'_{3}$, $V'_{4}$ and $V''_{4}$ of the non-central forces. The potential 
considered does not take account of the covalent interaction which leads to 
the chemical bonding. This (covalent) interaction can, however, be easily 
included in calculations by introducing additional terms of the type $V_{2}$, 
$V_{3}$ and $V_{4}$. 

As it is known harmonic non-central springs $V'_2$ are normally $5$ to $10$ 
times smaller than the central springs $V_2$. Our calculations of $V_3$ and 
$V'_3$ for alkali halides show that $V'_3/V_3$ is even smaller than 
$V'_2/V_2$. The same holds also for quartic and higher order anharmonic terms.
Therefore, as a first step, only central forces may 
be accounted. 

We calculate a strong local vibration of light impurity or host 
atom (ion) situated at the origin of our reference frame. In this case 
solutions of classical equations of motion, 
corresponding to the local mode, satisfy the conditions: 
$|A_{0}| \gg |A_{\vec{n}}|$. This allows one to suppose that the mode is well 
localized on the atom at the site $n=0$. Then in the approximation of central 
forces only coordinates of the central atom and of the directed to this atom 
components of coordinates of the nearest neighbor atoms contribute to the 
relaxation. We chose: 
${q}_1 = {x}_0$, ${q}_2 = {x}_{1x}$, ${q}_3 = {x}_{-1x}$, ${q}_4 = {y}_0$, 
${q}_5 = {y}_{1y}$, ${q}_6 = {y}_{-1y}$, ${q}_7 = {z}_0$, ${q}_8 = {z}_{1z}$, 
${q}_9 = {z}_{-1z}$. In this representation the impurity induces change of 
matrix $\bar{V}_2$; corresponding perturbation matrix is 
\[ V = \left (\begin{array}{ccc} 
w_x&0&0\\\noalign{\medskip} 
0&w_y&0\\\noalign{\medskip} 
0&0&w_z\end{array} 
\right ) , \;\; 
w_{\alpha} = \left (\begin{array}{ccc} 
\beta_{\alpha}&-\gamma_{\alpha}&-\gamma_{\alpha}\\\noalign{\medskip} 
-\gamma_{\alpha}&\gamma_{\alpha}&0\\\noalign{\medskip} 
-\gamma_{\alpha}&0&\gamma_{\alpha}\end{array} 
\right ) , \] 
($\alpha = x,y,z$). In harmonic approximation $\beta$ and $\gamma$ do 
not depend on $\alpha$: $\gamma = \Delta V_2$ is the change of the magnitude 
of central elastic constants due to defect atom,  
$\beta = 2\gamma+\omega^2 (1-M/M_1)$, $\omega$ is the frequency of the
normal 
mode, $M/M_1$ is the ratio of impurity and host central atom masses.
Quartic 
anharmonicity leads to amplitude dependent corrections of elastic constants 
and to their dependence of $\alpha$: 
\[
\beta_{\alpha} = \omega^2{\Big (}1- \frac{M}{M_1}{\Big )}+
2\gamma_{\alpha}, 
\]
\begin{equation}
\gamma_{\alpha} = \Delta V_2+\frac{1}{4}  A^2_{\alpha} {\Big \{ } V_4 +
\frac{V_3^2}{M_2} (G_{22}^{(0)}(0)-G_{23}^{(0)}(0)) {\Big \} }, 
\label{eq:gamma}
\end{equation} 
$A_{\alpha}$ is the $\alpha$'s Cartesian component of the amplitude of the 
local mode, $M_2$ is the mass of the nn to central ion, 
$G_{nn'}^{(0)}(0)$  are Greens functions of perfect lattice.
Here we neglect higher order terms with respect to $A^2$. 
This approximation holds if $ |A| < d A_0$, where $A_0=\sqrt{V_3/V_5}$, 
$d=1$
for ALM in (100)direction, $d=\sqrt{2}$ in (110) direction and $d=\sqrt{3}$
in (111) direction. In ${\rm NaI}$ $A_0 = 1.37 \AA$ in ${\rm KI}$ 
$A_0 = 1.36 \AA$, in ${\rm KCl}$ $A_0= 1.28 \AA$.
   
According to our calculations, in alkali halide crystals 
the second, negative term in the wavy brackets 
in (\ref{eq:gamma}) in its absolute value remarkably (typically $\sim 5$ 
times) exceeds the first term $V_4$. As a result, the term in the wavy 
brackets all-totally is negative. This means that in alkali halide crystals
ALMs can appear only in the gaps of the phonon spectrum. This conclusion
is in agreement with previous result of Kiselev and Sievers \cite{kiselev},
who showed that in ${\rm NaI}$  intrinsic local modes
can appear in the gap of the phonon spectrum 
if amplitude of {\rm Na) vibrations in (111) direction exceeds $0.4 \AA$.

Note that the softening essentially is the result of the long-range
relaxation of the positions of ions in the lattice. If one would allow only 
the nn ions to shift, then the second term  in the wavy brackets 
of Eq. (\ref{eq:gamma}) would be $- V_3^2/ V_2 \approx -V_4$ with few percent
accuracy and the full  factor would be small, either positive or negative.
  

To calculate ALMs we need to find the perturbed Green's functions  
\[ G = \left (\begin{array}{ccc} 
G_x&0&0\\ \noalign{\medskip} 
0&G_y&0\\ \noalign{\medskip} 
0&0&G_z\end{array}
\right ) , \;\;
G_{\alpha} = \left (\begin{array}{ccc} 
G_{\alpha 11}&G_{\alpha 12}&G_{\alpha 12}\\ \noalign{\medskip} 
G_{\alpha 12}&G_{\alpha 22}&0\\ \noalign{\medskip} 
G_{\alpha 12}&0&G_{\alpha 22}\end{array} 
\right ) , \] 
This can be done by applying the Lifshitz formula 
\[ G_{\alpha nn'} =
([I-G^{(0)}w_{\alpha}]^{-1}G^{(0)})_{nn'}\].


 
We performed calculation of ALMs associated with the strong 
local vibrations of $\{rm F}^-$, ${\rm{Cl}^-$ and ${\rm{Na}^-$ ions
in different alkali halide crystals with the gap in phonon spectrum,
for these ions being both, host and impure.  
The results are presented  in Figs 2 and 3.
The results for NaI are in agreement with direct 
calculations of vibrations of ${\rm Na}^-$ ion  in (111) direction in 
this lattice at large amplitudes \cite{kiselev}.

We also  we found that
at the critical amplitude $A_{cr} \geq 0.6 d \AA$r
a pole of the Greens functions at an small imaginary frequency appear,
which means that at  these  amplitudes the ALMs became unstable.
This is a natural result of the local softening of the lattice, which increases
fast with the amplitude $A$ (in our approximation as $A^2$). 
Note that $A_{cr}^2 \ll A_O^2$, which means that  
neglected higher-order anharmonic terms should not change this result.    

In conclusion, the  localized anharmonic modes in pure and impure crystals 
are considered. The applied method allows the reduction of nonlinear problem 
to a linear inverse problem of phonons scattering on a local potential. 
The methods is verified by consideration of            
intrinsic local  modes in monatomic chain. A simple analytical formula 
for the  softening of the
local springs by an odd anharmonic mode,  is obtain  and applied for  
calculation of anharmonic gap modes in pure and 
impure alkali halide crystals. It is found that at large amplitudes 
($\geq 1 \AA$ in (111) direction) the gap modes became unstable.
 


\section{Acknowledgment} 
This research was supported by the Estonian Science Foundation, Grant 
No.\~3864,$\ldots$.  D.N. thanks NORDITA for hospitality and financial support.
 
\begin{thebibliography}{99}

\bibitem{dolgov}
A.S.Dolgov, Fiz. Tverd. Tela (Leningrad) {\bf 28}, 1641 (1986)
[Sov. Phys. Solid tate {\bf 28}, 907 (1986)].
\bibitem{sivtak}
A.J.Sievers, S.Takeno, Phys. Rev. Lett. {\bf 61}, 970 (1988).
\bibitem{page}  
J.B.Page, Phys. Rev. B {\bf 41}, 7835 (1990).
\bibitem{kiselev}
S.A.Kiselev, S.R.Bickham, and A.J.Sievers, Phys. Rev. B {\bf 48},
13508 (1993)
\bibitem{zavt}
G.S.Zavt {\it et al.}, Phys. Rev. E {\bf 47}, 4108 (1993).
\bibitem{pagesievers}
J.B.Page, A.J.Sievers,$"Unusual Anharmonic Local Mode Systems"$,
in {\it Dynamical Properties of Solids}, G.K.Horton and 
A.A.Maradudin eds. (North Holland, Amsterdam, 1995),Vol. 5,  Ch. 3.
\bibitem{flach}
S.Flach, C.R.Willis, Physics Reports {\bf 295},181 (1998).
\bibitem{kisgap}
S.A.Kiselev, and A.J.Sievers, Phys. Rev. B {\bf 55}, 5755 (1997).
\bibitem{hizhrev}  
V.Hizhnyakov, Phys.Rev. B {\bf 53}, 13981 (1996); Europhys. Lett., 
{\bf 45} 508 (1999).  
\bibitem{hizhnev}  
V.Hizhnyakov and D.Nevedrov, Phys. Rev. B {\bf 56}, R2809 (1997). 
\bibitem{shell} 
A.D.B.Woods, W.Cochran, and B.N.Brockhouse,
Phys. Rev. {\bf 119}, 980 (1960);
A.D.Woods, B.N.Brockhouse, R.A.Cowley, and W.Cochran,
Phys. Rev {\bf 131}, 1025 (1963).
R.A.Cowley, W.Cochran, B.N.Brockhouse, and A.D.B.Woods,
Phys. Rev {\bf 131}, 1030 (1963).
\bibitem{bilz}
H.Bilz, W.Kress, {\it Phonon Dispersion Relations in
Insulators} (Springer, Berlin, 1979).
\bibitem{kristofel}
N.N.Kristofel, {\it Theoriya primesnyh centrov malyh radiusov
v ionnyh kristallah}  (Nauka, Moscow, 1974).
\bibitem{potential}
J.Michielsen, P.Woerlee, F.v.d.Graaf and J.A.A.Ketelaar, J. Chem. Soc. Faraday Trans. II {\bf 71}, 1730 (1975).
\bibitem{kosevich}
A.M.Kosevich and A.S.Kovalev, Zh. Eksp. Teor. Fiz. {\bf 67}, 1793 (1974)
[Sov. Phys. JETP {\bf 40}, 891 (1974)].
\bibitem{economou} 
E.N.Economou, {\it Green's Functions in Quantum Physics} 
(Springer-Verlag, Berlin, 1983). 
\bibitem{maradudin}  
A.A.Maradudin {\it et al.} {\it Theory of Lattice Dynamics in  
Harmonic Approximation} (Academic, New York, 1963);   
A.A.Maradudin, {\it Theoretical and 
Experimental Uspects of the Effects of Point Defects and Disorder of the  
Vibrations of Crystals} (Academic, New York 1966).

\end{thebibliography} 

\bigskip

\begin{center}
{\Large {\bf Figure captions}}
\end{center}

\bigskip

\noindent
{\bf Figure 1.} Anharmonic local modes in monatomic chain with 
quartic anharmonicity; 
$\beta$ (solid lines) and $A_l$ (dot-dashed lines) vs. 
$\omega_l$ for even and odd modes. We used for calculations 
$\omega_M=1$ and $V^{(4)}=1$.

\noindent
{\bf Figure 2.} Frequency of the impurity anharmonic gap mode 
($w_l/w_m$) vs. amplitude (in $\AA$). a) ${\rm KI:Na}$, 
b) ${\rm KI:Cl}$; [100] -- solid line, [110] -- dashed line,
[111] -- dot-dashed line.

\noindent
{\bf Figure 3.}  Frequency of the anharmonic gap mode  ($w_l/w_m$) vs.
amplitude (in $\AA$) in pure lattices; a) ${\rm NaI}$, b) ${\rm KI}$,
c)  ${\rm NaBr}$  and ${\rm RbF}$; 
[100] -- solid line, [110] -- dashed line,
[111] -- dot-dashed line.

\end{document} 
















